% ------------------------------------- 
% SuplementaryAllBiomarkers 
% ------------------------------------- 

\begin{table}[H]
    \centering

\resizebox {0.95\textwidth}{!}{ 

 	\renewcommand{\arraystretch}{1.5} 
    \begin{tabular}{ lllllll }
        \hline
        \rowcolor[HTML]{ FFFFC7  }

         \textbf{ Acronym } &         \textbf{ Protein } &         \textbf{ UniProt } &         \textbf{ LOD$\_$Batch$\_$20160383 } &         \textbf{ LOD$\_$Batch$\_$20160977 } &         \textbf{ Uniprt$\_$Web } &         \textbf{ Wiki$\_$Web } \\ 
        \hline 

        \multicolumn{1}{l|}{ ADA } &  Adenosine Deaminase   & P00813   & 0.436494   & 1.584419   & \url{http://www.uniprot.org/uniprot/P00813}   & \url{https://en.wikipedia.org/wiki/Adenosine \textunderscore deaminase}          \\ 
        \multicolumn{1}{l|}{ ARTN } &  Artemin   & Q5T4W7   & 0.031349   & 0.031349   & \url{http://www.uniprot.org/uniprot/Q5T4W7}   & \url{https://en.wikipedia.org/wiki/Artemin}          \\ 
        \multicolumn{1}{l|}{ AXIN1 } &  Axin-1   & O15169   & 0.845030   & 0.576816   & \url{http://www.uniprot.org/uniprot/O15169}   & \url{https://en.wikipedia.org/wiki/AXIN1}          \\ 
        \multicolumn{1}{l|}{ BDNF } &  Brain-derived neurotrophic factor   & P23560   & -0.380273   & -0.045445   & \url{http://www.uniprot.org/uniprot/P23560}   & \url{https://en.wikipedia.org/wiki/Brain-derived \textunderscore neurotrophic \textunderscore factor}          \\ 
        \multicolumn{1}{l|}{ BNGF } &  Beta-nerve growth factor   & P01138   & 0.755167   & 0.631771   & \url{http://www.uniprot.org/uniprot/P01138}   &           \\ 
        \multicolumn{1}{l|}{ CASP8 } &  Caspase-8   & Q14790   & 0.507711   & 0.151261   & \url{http://www.uniprot.org/uniprot/Q14790}   & \url{https://en.wikipedia.org/wiki/Caspase \textunderscore 8}          \\ 
        \multicolumn{1}{l|}{ CCL11 } &  Eotaxin   & P51671   & 1.427776   & 0.950032   & \url{http://www.uniprot.org/uniprot/P51671}   & \url{https://en.wikipedia.org/wiki/CCL11}          \\ 
        \multicolumn{1}{l|}{ CCL19 } &  C-C motif chemokine 19   & Q99731   & 0.988040   & -0.038600   & \url{http://www.uniprot.org/uniprot/Q99731}   & \url{https://en.wikipedia.org/wiki/CCL19}          \\ 
        \multicolumn{1}{l|}{ CCL20 } &  C-C motif chemokine 20   & P78556   & 1.276281   & 1.290873   & \url{http://www.uniprot.org/uniprot/P78556}   & \url{https://en.wikipedia.org/wiki/CCL20}          \\ 
        \multicolumn{1}{l|}{ CCL23 } &  C-C motif chemokine 23   & P55773   & 0.780150   & 0.047888   & \url{http://www.uniprot.org/uniprot/P55773}   & \url{https://en.wikipedia.org/wiki/CCL23}          \\ 
        \multicolumn{1}{l|}{ CCL25 } &  C-C motif chemokine 25   & O15444   & 1.083723   & 0.634603   & \url{http://www.uniprot.org/uniprot/O15444}   & \url{https://en.wikipedia.org/wiki/CCL25}          \\ 
        \multicolumn{1}{l|}{ CCL28 } &  C-C motif chemokine 28   & Q9NRJ3   & 0.069990   & -0.046866   & \url{http://www.uniprot.org/uniprot/Q9NRJ3}   & \url{https://en.wikipedia.org/wiki/CCL28}          \\ 
        \multicolumn{1}{l|}{ CCL3 } &  C-C motif chemokine 3   & P10147   & -0.077074   & -0.524618   & \url{http://www.uniprot.org/uniprot/P10147}   & \url{https://en.wikipedia.org/wiki/CCL3}          \\ 
        \multicolumn{1}{l|}{ CCL4 } &  C-C motif chemokine 4   & P13236   & 0.392063   & -0.121811   & \url{http://www.uniprot.org/uniprot/P13236}   & \url{https://en.wikipedia.org/wiki/CCL4}          \\ 
        \multicolumn{1}{l|}{ CD244 } &  Natural killer cell receptor 2B4   & Q9BZW8   & 1.658169   & 1.062742   & \url{http://www.uniprot.org/uniprot/Q9BZW8}   & \url{https://en.wikipedia.org/wiki/CD244}          \\ 
        \multicolumn{1}{l|}{ CD40 } &  CD40L receptor   & P25942   & 0.757131   & -0.447591   & \url{http://www.uniprot.org/uniprot/P25942}   & \url{https://en.wikipedia.org/wiki/CD40 \textunderscore (protein)}          \\ 
        \multicolumn{1}{l|}{ CD5 } &  T-cell surface glycoprotein CD5   & P06127   & -0.487334   & -0.578852   & \url{http://www.uniprot.org/uniprot/P06127}   & \url{https://en.wikipedia.org/wiki/CD5 \textunderscore (protein)}          \\ 
        \multicolumn{1}{l|}{ CD6 } &  T cell surface glycoprotein CD6 isoform   & Q8WWJ7   & -0.194972   & -0.146330   & \url{http://www.uniprot.org/uniprot/Q8WWJ7}   & \url{https://en.wikipedia.org/wiki/CD6}          \\ 
        \multicolumn{1}{l|}{ CDCP1 } &  CUB domain-containing protein 1   & Q9H5V8   & 0.367527   & 0.038621   & \url{http://www.uniprot.org/uniprot/Q9H5V8}   & \url{https://en.wikipedia.org/wiki/CDCP1}          \\ 
        \multicolumn{1}{l|}{ CSF1 } &  Macrophage colony-stimulating factor 1   & P09603   & -0.003590   & 0.396328   & \url{http://www.uniprot.org/uniprot/P09603}   & \url{https://en.wikipedia.org/wiki/Macrophage \textunderscore colony-stimulating \textunderscore factor}          \\ 
        \multicolumn{1}{l|}{ CST5 } &  Cystatin D   & P28325   & 0.046105   & 5.808007   & \url{http://www.uniprot.org/uniprot/P28325}   & \url{https://en.wikipedia.org/wiki/CST5}          \\ 
        \multicolumn{1}{l|}{ CX3CL1 } &  Fractalkine   & P78423   & 1.875148   & 1.166002   & \url{http://www.uniprot.org/uniprot/P78423}   & \url{https://en.wikipedia.org/wiki/CX3CL1}          \\ 
        \multicolumn{1}{l|}{ CXCL1 } &  C-X-C motif chemokine 1   & P09341   & 1.387787   & 0.758507   & \url{http://www.uniprot.org/uniprot/P09341}   & \url{https://en.wikipedia.org/wiki/CXCL10}          \\ 
        \multicolumn{1}{l|}{ CXCL10 } &  C-X-C motif chemokine 10   & P02778   & 1.534295   & 1.358654   & \url{http://www.uniprot.org/uniprot/P02778}   & \url{https://en.wikipedia.org/wiki/CXCL10}          \\ 
        \multicolumn{1}{l|}{ CXCL11 } &  C-X-C motif chemokine 11   & O14625   & 1.471448   & 0.111323   & \url{http://www.uniprot.org/uniprot/O14625}   & \url{https://en.wikipedia.org/wiki/CXCL11}          \\ 
        \multicolumn{1}{l|}{ CXCL5 } &  C-X-C motif chemokine 5   & P42830   & 1.184377   & 1.639521   & \url{http://www.uniprot.org/uniprot/P42830}   & \url{https://en.wikipedia.org/wiki/CXCL5}          \\ 
        \multicolumn{1}{l|}{ CXCL6 } &  C-X-C motif chemokine 6   & P80162   & 0.843005   & 0.398682   & \url{http://www.uniprot.org/uniprot/P80162}   & \url{https://en.wikipedia.org/wiki/CXCL6}          \\ 
        \multicolumn{1}{l|}{ CXCL9 } &  C-X-C motif chemokine 9   & Q07325   & 1.559012   & 1.430370   & \url{http://www.uniprot.org/uniprot/Q07325}   & \url{https://en.wikipedia.org/wiki/CXCL9}          \\ 
        \multicolumn{1}{l|}{ DNER } &  Delta and Notch-like epidermal growth factor-related receptor   & Q8NFT8   & -0.127219   & -0.730436   & \url{http://www.uniprot.org/uniprot/Q8NFT8}   & \url{https://en.wikipedia.org/wiki/DNER}          \\ 
        \multicolumn{1}{l|}{ EIF4EBP1 } &  Eukaryotic translation initiation factor 4E-binding protein 1   & Q13541   & 0.893928   & 0.969980   & \url{http://www.uniprot.org/uniprot/Q13541}   & \url{https://en.wikipedia.org/wiki/EIF4EBP1}          \\ 
        \multicolumn{1}{l|}{ ENRAGE } &  Protein S100-A12   & P80511   & 0.313350   & 0.996331   & \url{http://www.uniprot.org/uniprot/P80511}   & \url{https://en.wikipedia.org/wiki/S100A12}          \\ 
        \multicolumn{1}{l|}{ FGF19 } &  Fibroblast growth factor 19   & O95750   & 0.662450   & 0.255022   & \url{http://www.uniprot.org/uniprot/O95750}   & \url{https://en.wikipedia.org/wiki/FGF19}          \\ 
        \multicolumn{1}{l|}{ FGF21 } &  Fibroblast growth factor 21   & Q9NSA1   & 0.844435   & -0.310457   & \url{http://www.uniprot.org/uniprot/Q9NSA1}   & \url{https://en.wikipedia.org/wiki/FGF21}          \\ 
        \multicolumn{1}{l|}{ FGF23 } &  Fibroblast growth factor 23   & Q9GZV9   & 1.039348   & 1.108382   & \url{http://www.uniprot.org/uniprot/Q9GZV9}   & \url{https://en.wikipedia.org/wiki/FGF23}          \\ 
        \multicolumn{1}{l|}{ FGF5 } &  Fibroblast growth factor 5   & Q8NF90   & 1.142597   & 0.876939   & \url{http://www.uniprot.org/uniprot/Q8NF90}   & \url{https://en.wikipedia.org/wiki/FGF5}          \\ 
        \multicolumn{1}{l|}{ FLT3L } &  Fms-related tyrosine kinase 3 ligand   & P49771   & 1.866726   & 1.119030   & \url{http://www.uniprot.org/uniprot/P49771}   & \url{https://en.wikipedia.org/wiki/FLT3LG}          \\ 
        \multicolumn{1}{l|}{ GDNF } &  Glial cell line-derived neurotrophic factor   & P39905   & 1.331378   & 1.648532   & \url{http://www.uniprot.org/uniprot/P39905}   & \url{https://en.wikipedia.org/wiki/Glial \textunderscore cell \textunderscore line-derived \textunderscore neurotrophic \textunderscore factor}          \\ 
        \multicolumn{1}{l|}{ HGF } &  Hepatocyte growth factor   & P14210   & 1.146276   & 0.395915   & \url{http://www.uniprot.org/uniprot/P14210}   & \url{https://en.wikipedia.org/wiki/Hepatocyte \textunderscore growth \textunderscore factor}          \\ 
        \multicolumn{1}{l|}{ IFNG } &  Interferon gamma   & P01579   & 0.992133   & 0.992133   & \url{http://www.uniprot.org/uniprot/P01579}   & \url{https://en.wikipedia.org/wiki/Interferon \textunderscore gamma}          \\ 
        \multicolumn{1}{l|}{ IL10 } &  Interleukin-10   & P22301   & 1.839415   & 2.432488   & \url{http://www.uniprot.org/uniprot/P22301}   & \url{https://en.wikipedia.org/wiki/Interleukin \textunderscore 10}          \\ 
        \multicolumn{1}{l|}{ IL10RA } &  Interleukin-10 receptor subunit alpha   & Q13651   & 0.996689   & 0.662247   & \url{http://www.uniprot.org/uniprot/Q13651}   & \url{https://en.wikipedia.org/wiki/Interleukin \textunderscore 10 \textunderscore receptor, \textunderscore alpha \textunderscore subunit}          \\ 
        \multicolumn{1}{l|}{ IL10RB } &  Interleukin-10 receptor subunit beta   & Q08334   & 1.425411   & 1.405083   & \url{http://www.uniprot.org/uniprot/Q08334}   & \url{https://en.wikipedia.org/wiki/Interleukin \textunderscore 10 \textunderscore receptor, \textunderscore beta \textunderscore subunit}          \\ 
        \multicolumn{1}{l|}{ IL12B } &  Interleukin-12 subunit beta   & P29460   & -0.338237   & -0.143724   & \url{http://www.uniprot.org/uniprot/P29460}   & \url{https://en.wikipedia.org/wiki/Interleukin \textunderscore 12 \textunderscore receptor, \textunderscore beta \textunderscore 1 \textunderscore subunit}          \\ 
        \multicolumn{1}{l|}{ IL13 } &  Interleukin-13   & P35225   & 1.537823   & 1.537823   & \url{http://www.uniprot.org/uniprot/P35225}   & \url{https://en.wikipedia.org/wiki/Interleukin \textunderscore 13}          \\ 
        \multicolumn{1}{l|}{ IL15RA } &  Interleukin-15 receptor subunit alpha   & Q13261   & 0.783341   & 0.595480   & \url{http://www.uniprot.org/uniprot/Q13261}   & \url{https://en.wikipedia.org/wiki/Interleukin \textunderscore 15 \textunderscore receptor, \textunderscore alpha \textunderscore subunit}          \\ 
        \multicolumn{1}{l|}{ IL17A } &  Interleukin-17A   & Q16552   & 0.532945   & 0.371852   & \url{http://www.uniprot.org/uniprot/Q16552}   & \url{https://en.wikipedia.org/wiki/IL17A}          \\ 
        \multicolumn{1}{l|}{ IL17C } &  Interleukin-17C   & Q9P0M4   & 1.371362   & 1.358013   & \url{http://www.uniprot.org/uniprot/Q9P0M4}   &           \\ 
        \multicolumn{1}{l|}{ IL18 } &  Interleukin-18   & Q14116   & -0.188372   & 0.365590   & \url{http://www.uniprot.org/uniprot/Q14116}   & \url{https://en.wikipedia.org/wiki/Interleukin \textunderscore 18}          \\ 
        \multicolumn{1}{l|}{ IL18R1 } &  Interleukin-18 receptor 1   & Q13478   & 0.933131   & 0.638867   & \url{http://www.uniprot.org/uniprot/Q13478}   & \url{https://en.wikipedia.org/wiki/Interleukin-18 \textunderscore receptor}          \\ 
        \multicolumn{1}{l|}{ IL1A } &  Interleukin-1 alpha   & P01583   & 0.336995   & 1.802489   & \url{http://www.uniprot.org/uniprot/P01583}   & \url{https://en.wikipedia.org/wiki/IL1A}          \\ 
        \multicolumn{1}{l|}{ IL2 } &  Interleukin-2   & P60568   & 1.223237   & 1.223237   & \url{http://www.uniprot.org/uniprot/P60568}   & \url{https://en.wikipedia.org/wiki/Interleukin \textunderscore 2}          \\ 
        \multicolumn{1}{l|}{ IL20 } &  Interleukin-20   & Q9NYY1   & 0.728374   & 0.813528   & \url{http://www.uniprot.org/uniprot/Q9NYY1}   & \url{https://en.wikipedia.org/wiki/Interleukin \textunderscore 20}          \\ 
        \multicolumn{1}{l|}{ IL20RA } &  Interleukin-20 receptor subunit alpha   & Q9UHF4   & 0.877718   & 0.881812   & \url{http://www.uniprot.org/uniprot/Q9UHF4}   &           \\ 
        \multicolumn{1}{l|}{ IL22RA1 } &  Interleukin-22 receptor subunit alpha-1   & Q8N6P7   & 2.260242   & 2.260242   & \url{http://www.uniprot.org/uniprot/Q8N6P7}   &           \\ 
        \multicolumn{1}{l|}{ IL24 } &  Interleukin-24   & Q13007   & 1.336190   & 1.336190   & \url{http://www.uniprot.org/uniprot/Q13007}   & \url{https://en.wikipedia.org/wiki/Interleukin \textunderscore 24}          \\ 
        \multicolumn{1}{l|}{ IL2RB } &  Interleukin-2 receptor subunit beta   & P14784   & 0.845790   & 0.845790   & \url{http://www.uniprot.org/uniprot/P14784}   & \url{https://en.wikipedia.org/wiki/IL2RB}          \\ 
        \multicolumn{1}{l|}{ IL33 } &  Interleukin-33   & O95760   & 1.425509   & 1.425509   & \url{http://www.uniprot.org/uniprot/O95760}   & \url{https://en.wikipedia.org/wiki/Interleukin \textunderscore 33}          \\ 
        \multicolumn{1}{l|}{ IL4 } &  Interleukin-4   & P05112   & 1.184842   & 0.958605   & \url{http://www.uniprot.org/uniprot/P05112}   & \url{https://en.wikipedia.org/wiki/Interleukin \textunderscore 4}          \\ 
        \multicolumn{1}{l|}{ IL5 } &  Interleukin-5   & P05113   & 1.725314   & 1.647055   & \url{http://www.uniprot.org/uniprot/P05113}   & \url{https://en.wikipedia.org/wiki/Interleukin \textunderscore 5}          \\ 
        \multicolumn{1}{l|}{ IL6 } &  Interleukin-6   & P05231   & 0.824445   & 2.415735   & \url{http://www.uniprot.org/uniprot/P05231}   & \url{https://en.wikipedia.org/wiki/Interleukin \textunderscore 6}          \\ 
        \multicolumn{1}{l|}{ IL7 } &  Interleukin-7   & P13232   & 1.021735   & 1.336047   & \url{http://www.uniprot.org/uniprot/P13232}   & \url{https://en.wikipedia.org/wiki/Interleukin \textunderscore 7}          \\ 
        \multicolumn{1}{l|}{ IL8 } &  Interleukin-8   & P10145   & 1.162271   & 2.227435   & \url{http://www.uniprot.org/uniprot/P10145}   & \url{https://en.wikipedia.org/wiki/Interleukin \textunderscore 8}          \\ 
        \multicolumn{1}{l|}{ LIF } &  Leukemia inhibitory factor   & P15018   & 0.800844   & 0.800844   & \url{http://www.uniprot.org/uniprot/P15018}   & \url{https://en.wikipedia.org/wiki/Leukemia \textunderscore inhibitory \textunderscore factor}          \\ 
        \multicolumn{1}{l|}{ LIFR } &  Leukemia inhibitory factor receptor   & P42702   & 1.665534   & -0.265929   & \url{http://www.uniprot.org/uniprot/P42702}   & \url{https://en.wikipedia.org/wiki/LIFR}          \\ 
        \multicolumn{1}{l|}{ MCP1 } &  Monocyte chemotactic protein 1   & P13500   & 0.358877   & -0.161967   & \url{http://www.uniprot.org/uniprot/P13500}   & \url{https://en.wikipedia.org/wiki/Monocyte \textunderscore chemoattractant \textunderscore protein \textunderscore 1}          \\ 
        \multicolumn{1}{l|}{ MCP2 } &  Monocyte chemotactic protein 2   & P80075   & 1.385177   & 1.823898   & \url{http://www.uniprot.org/uniprot/P80075}   &           \\ 
        \multicolumn{1}{l|}{ MCP3 } &  Monocyte chemotactic protein 3   & P80098   & 1.493173   & 1.699734   & \url{http://www.uniprot.org/uniprot/P80098}   &           \\ 
        \multicolumn{1}{l|}{ MCP4 } &  Monocyte chemotactic protein 4   & Q99616   & -0.265469   & -0.298464   & \url{http://www.uniprot.org/uniprot/Q99616}   &           \\ 
        \multicolumn{1}{l|}{ MMP1 } &  Matrix metalloproteinase-1   & P03956   & -0.024189   & -6.622735   & \url{http://www.uniprot.org/uniprot/P03956}   & \url{https://en.wikipedia.org/wiki/Matrix \textunderscore metalloproteinase}          \\ 
        \multicolumn{1}{l|}{ MMP10 } &  Matrix metalloproteinase-10   & P09238   & 1.379258   & 3.725904   & \url{http://www.uniprot.org/uniprot/P09238}   & \url{https://en.wikipedia.org/wiki/Matrix \textunderscore metalloproteinase}          \\ 
        \multicolumn{1}{l|}{ NRTN } &  Neurturin   & Q99748   & 1.124936   & 1.124936   & \url{http://www.uniprot.org/uniprot/Q99748}   & \url{https://en.wikipedia.org/wiki/Neurturin}          \\ 
        \multicolumn{1}{l|}{ NT3 } &  Neurotrophin-3   & P20783   & 0.771270   & 0.918843   & \url{http://www.uniprot.org/uniprot/P20783}   & \url{https://en.wikipedia.org/wiki/Neurotrophin-3}          \\ 
        \multicolumn{1}{l|}{ OPG } &  Osteoprotegerin   & O00300   & 0.918419   & 0.590118   & \url{http://www.uniprot.org/uniprot/O00300}   & \url{https://en.wikipedia.org/wiki/Osteoprotegerin}          \\ 
        \multicolumn{1}{l|}{ OSM } &  Oncostatin-M   & P13725   & -0.153103   & -0.025163   & \url{http://www.uniprot.org/uniprot/P13725}   & \url{https://en.wikipedia.org/wiki/Oncostatin \textunderscore M}          \\ 
        \multicolumn{1}{l|}{ PDL1 } &  Programmed cell death 1 ligand 1   & Q9NZQ7   & 2.257393   & 2.092503   & \url{http://www.uniprot.org/uniprot/Q9NZQ7}   & \url{https://en.wikipedia.org/wiki/PD-L1}          \\ 
        \multicolumn{1}{l|}{ SCF } &  Stem cell factor   & P21583   & 0.922578   & 0.051798   & \url{http://www.uniprot.org/uniprot/P21583}   & \url{https://en.wikipedia.org/wiki/Stem \textunderscore cell \textunderscore factor}          \\ 
        \multicolumn{1}{l|}{ SIRT2 } &  SIR2-like protein 2   & Q8IXJ6   & 1.402289   & 1.386472   & \url{http://www.uniprot.org/uniprot/Q8IXJ6}   &           \\ 
        \multicolumn{1}{l|}{ SLAMF1 } &  Signaling lymphocytic activation molecule   & Q13291   & 1.849931   & 1.677337   & \url{http://www.uniprot.org/uniprot/Q13291}   & \url{https://en.wikipedia.org/wiki/Signaling \textunderscore lymphocytic \textunderscore activation \textunderscore molecule}          \\ 
        \multicolumn{1}{l|}{ ST1A1 } &  Sulfotransferase 1A1   & P50225   & 0.078597   & 0.568043   & \url{http://www.uniprot.org/uniprot/P50225}   & \url{https://en.wikipedia.org/wiki/SULT1A1}          \\ 
        \multicolumn{1}{l|}{ STAMBP } &  STAM-binding protein   & O95630   & 0.667136   & 0.627816   & \url{http://www.uniprot.org/uniprot/O95630}   & \url{https://en.wikipedia.org/wiki/STAMBP}          \\ 
        \multicolumn{1}{l|}{ TGFA } &  Transforming growth factor alpha   & P01135   & -1.214780   & -1.869967   & \url{http://www.uniprot.org/uniprot/P01135}   & \url{https://en.wikipedia.org/wiki/TGF \textunderscore alpha}          \\ 
        \multicolumn{1}{l|}{ TGFB1 } &  Latency-associated peptide transforming growth factor beta-1   & P01137   & 1.034369   & 0.482168   & \url{http://www.uniprot.org/uniprot/P01137}   & \url{https://en.wikipedia.org/wiki/TGF \textunderscore beta \textunderscore 1}          \\ 
        \multicolumn{1}{l|}{ TNF } &  Tumor necrosis factor   & P01375   & 0.831819   & 0.837656   & \url{http://www.uniprot.org/uniprot/P01375}   & \url{https://en.wikipedia.org/wiki/Tumor \textunderscore necrosis \textunderscore factor}          \\ 
        \multicolumn{1}{l|}{ TNFB } &  TNF-beta   & P01374   & 0.605630   & 0.200990   & \url{http://www.uniprot.org/uniprot/P01374}   & \url{https://en.wikipedia.org/wiki/Lymphotoxin \textunderscore alpha}          \\ 
        \multicolumn{1}{l|}{ TNFRSF9 } &  Tumor necrosis factor receptor superfamily member 9   & Q07011   & 1.599546   & 1.466786   & \url{http://www.uniprot.org/uniprot/Q07011}   & \url{https://en.wikipedia.org/wiki/4-1BB \textunderscore ligand}          \\ 
        \multicolumn{1}{l|}{ TNFSF14 } &  Tumor necrosis factor ligand superfamily member 14   & O43557   & 0.210933   & -0.170624   & \url{http://www.uniprot.org/uniprot/O43557}   & \url{https://en.wikipedia.org/wiki/LIGHT \textunderscore (protein)}          \\ 
        \multicolumn{1}{l|}{ TRAIL } &  TNF-related apoptosis-inducing ligand   & P50591   & 0.651508   & 0.548601   & \url{http://www.uniprot.org/uniprot/P50591}   & \url{https://en.wikipedia.org/wiki/TRAIL}          \\ 
        \multicolumn{1}{l|}{ TRANCE } &  TNF-related activation-induced cytokine   & O14788   & 1.263670   & 1.118725   & \url{http://www.uniprot.org/uniprot/O14788}   & \url{https://en.wikipedia.org/wiki/Receptor \textunderscore activator \textunderscore of \textunderscore nuclear \textunderscore factor \textunderscore kappa-B \textunderscore ligand}          \\ 
        \multicolumn{1}{l|}{ TSLP } &  Thymic stromal lymphopoietin   & Q969D9   & 1.080835   & 1.080835   & \url{http://www.uniprot.org/uniprot/Q969D9}   & \url{https://en.wikipedia.org/wiki/Thymic \textunderscore stromal \textunderscore lymphopoietin}          \\ 
        \multicolumn{1}{l|}{ TWEAK } &  Tumor necrosis factor   & O43508   & 0.511139   & 0.439180   & \url{http://www.uniprot.org/uniprot/O43508}   & \url{https://en.wikipedia.org/wiki/Tumor \textunderscore necrosis \textunderscore factor}          \\ 
        \multicolumn{1}{l|}{ UPA } &  Urokinase-type plasminogen activator   & P00749   & 0.767444   & 0.691054   & \url{http://www.uniprot.org/uniprot/P00749}   & \url{https://en.wikipedia.org/wiki/Urokinase}          \\ 
        \multicolumn{1}{l|}{ VEGFA } &  Vascular endothelial growth factor A   & P15692   & 1.566666   & 1.233603   & \url{http://www.uniprot.org/uniprot/P15692}   & \url{https://en.wikipedia.org/wiki/Vascular \textunderscore endothelial \textunderscore growth \textunderscore factor}          \\ 

    \end{tabular}
 } 
 
    \caption{Summary of all biomarkers. From left to right, a short acronym with the protein ID, protein name, UniProt ID, LOD value for each of the two-run batches, UniProt web with the protein, and Wikipedia link with the protein.}
    \label{table:SuplementaryAllBiomarkers}

\end{table}

% -------------------------------------  
% Notice that you need to compile twice  
% the latexPDF in order for the index of 
% figures and tables to work properly    
% -------------------------------------  
