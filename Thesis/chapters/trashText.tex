


\section{Introduction}


\section{Vitamin D metabolism }

Vitamin D, along with vitamins A, E and K, belong to the fat-soluble group, they are absorbed through the intestinal tract inside lipids droplets. Vitamin D can be stored inside liver’s fat tissue for about 2 months and be released on demand. There are two methods by which the body obtains Vitamin D, via sun exposure (pre-vitamin D3) and food intake or dietary supplements (vitamin D2 and D3) in the small intestine via passive diffusion using intestinal membrane carrier proteins [D8]. 

Regardless of intake method, vitamin D needs to undergo two hydroxylation processes. The first one occurs in the liver and transforms vitamin D into 25-hydroxyvitamin D, 25(OH)D, using 25-hydroxylase. The second hydroxylation happens in the kidneys and forms 1,25-dihydroxyvitamin D ,1,25(OH)2D, using 1-α-hydroxylase, which is the final and physiologically active form of vitamin D. This form is known as calcitriol. If calcitriol becomes excessive, then it is converted to 24,25-dihydroxycholecalciferiol, which is less active.  

Both vitamins D2 (ergocalciferol) and D3 (cholecalciferol) raise 25(OH)D levels, the metabolism and actions of vitamins D2 and D3 are identical, and they only differ in the chemical morphology present in their sidechain structure. Evidence suggests that vitamin D3 increases 25(OH)D levels greater and longer than vitamin D2 [D25]. The other 3 forms vitamin D, D1, D4, and D5 are not relevant in this study. 

There are two primary functions for vitamin D, to absorb calcium and phosphate from the guts into the blood; and to inhibit parathyroid hormone (PTH) production. When calcitriol is present in the blood, it stimulates the epithelial cells in the gut to increase production of calbindin-D proteins. This increases the absorption of calcium from the brush border to the basolateral membrane, where calcium finally enters the bloodstream. 


\section{PTH and Calcitonin}

PTH maintains healthy levels of calcium in the blood serum. If calcium levels are too low, then PTH binds to osteoblast, which increases RANKL, which transforms pre-osteoclasts into osteoclasts. Osteoclast breaks down bones apart in order to maintain calcium ions levels as they should be. PTH also increases renal re-absorption in the distal convoluted tubule (DCT). Ultimately, this calcium is reabsorbed into the blood, but in exchange, you will urinate more phosphate. PTH also increases the production of vitamin D in the kidney, by increasing the production of 1-α-hydroxylase. PTH also inhibits the production of PTH, making a negative feedback loop. Malignant tumors can also secrete PTH peptides (PTHrP) that mimic PTH which help tumors grow by increasing calcium levels. PTH can also lead to the increase of IL1, IL2, IL6 and INF-ɣ , which increases osteoclastogenesis that leads to further bone density loss [(Zanchetta et al., 2016; Micic et al., 2019)]. 

Calcitonin is secreted in the thyroid glands. Calcitonin simply decreases calcium levels by inhibiting osteoclasts. Having high calcium levels in the blood serum is not as dangerous compared to having too little, thus calcitonin has less homeostasis power than PTH and calcitriol. 


\section{25OHD standardization}

Antibodies and chromatography are the two main methods to assess 25OHD levels. Each has several sub-methods that give high variability in the final 25OHD levels [D12, D13, D14]. As a result, comparing studies is challenging. The aim of VDSP is to create a global partnership that aims to standardize the laboratory assessment of vitamin D levels. By ensuring that the serum total 25OHD measurements are uniform and precise, the program seeks to enhance the identification, assessment, and management of vitamin D insufficiency and deficiency across different regions, times, and laboratory methodologies [VDSP] 



\section{Vitamin D sources}


UVB radiation 

 

UVB radiation can come from the Sun or tanning beds. It has several detrimental health effects, including "thymine dimer" DNA damage with the subsequent sunburn, hydroxyl and oxygen radicals, melanomas due melanocytes DNA damage, vitamin A depletion due to keratinocytes and melanocytes DNA damage. UVB always comes along with UVA radiation in about 5 to 95 proportion; UVA offers no health advantages, and it has more penetration power. UVA causes collagen breakdown, free radicals, vitamin A depletion, cataracts, and suppression of T lymphocytes and apoptosis which results in the proliferation of melanoma, basal cell carcinoma, and squamous cell carcinoma.  Sunscreen can protect against UVA and UVB, however the sun protection factor (SPF) only describes UVB blocking, and the lotion can be deceivingly marketed as "Sun protection" when in reality it offers no UVA protection at all. There is no standard for UVA protection labeling and even the SPF standard changes from country to country. Darker skin lowers UVB absorption [D9]. Skin color is mostly determined by melanine, and slightly by carotenes and iron levels in blood. Melanine is determined by genetics and increases slightly by the amount of UVB exposure. 

 

Food 

 

Foods with natural source of vitamin D are almost exclusively concentrated in the fat fish group. Namely mackerel (1010 IU D2+D3/100gr, 168% RDA [D66]), raw salmon (435 IU, 73%) or caviar (117 IU , 20%). White fish has very little amount in comparison, such as cod (36 IU, 6%), however fish composite where the whole fish is mixed including liver or kidneys, has increased amounts such as canned tuna (269 IU, 45%). Meat and poultry also have a small amount, beef (28 IU, 5%), bacon (9 IU, 2%) or chicken (9 IU, 2%). Dairy is a mixed bag; egg yolk is rich in fatty acids and contain high levels, while the white part does not, in average a whole raw egg contain about 91 IU D2+D3/100gr, 15% RDA; while whole 3.5% milk has near zero (2 IU, 0%). Vegetables, fungi, fruits and nuts also contain no natural vitamin D. However, mushrooms grown under artificially high UV light contain about 1140 IU D2+D3/100gr, 190% RDA. [D1] Food preparation decreases by about 10% the amount of vitamin D contained in their raw format counterpart. 

 

[[Furthermore, in the last 70 years food trends have changed from healthy fresh meals for quick processed frozen food with high caloric input and low nutritional value.]] 

 

Supplements 

 

Vitamin D supplements are widely available in supermarkets and pharmacies as over the counter drugs; these versions already contain more than the daily recommendation doses of vitamin D. Higher concentration pills requires medical prescription. Hypervitaminosis D is almost exclusively occurring due high intake of dietary supplements [D45-D47] 

 

Diets that avoid animal-based products require supplementation. However, some supplements are also animal based. Vitamin D2 is obtained by irradiating ergosterol in yeast, while vitamin D3 can be obtained with mushroom powder, irradiating pre-vitamin D3 from lanolin which comes from the wool of sheep, using a similar process with lichen instead, or directly extracting oil from fish liver. 

 

Epidemiology 

 

There is an ongoing shift in demographics in first world countries that are predominantly of Caucasian ethnicity with people of middle easter, black, and south Asian sea descent who require higher UVB to synthesize vitamin D [1] [3] [Meta-Migrant low D].  The prevalence of vitamin D deficiency (25OHD < 50nmol/l) is significantly elevated around the world [D33-D36]. Tromsø (19% prevalence) have a low prevalence compared to Norway (28%), which also have a lower prevalence compared to Europe (40%) and other European countries: Greece (62%), Germany (44%), Netherlands ( general population 29%, 68% in male and 53% in female Chinese immigrants [Chinese Netherland]), Ireland (27%), UK (57%, 94% among Bangladeshi background [BangladeshMigrant]), Denmark (general population 37%, foreigners in clinical setting 80% [DanishMigrant]), Finland (64% foreigners, 7% nationals), Iceland (34%), Estonia (29%) [Estonia].   Italy ( 77.5% foreign children [ItalyMigrantChildren] ) Switzerland ( 86% foreigners [SwissMigrant]) . In other parts of the world, we have Canada (37%), USA (24%), Australia (23%), in the Asia region (58%) or Africa region (34%). 

 

Vitamin D has been related to several diseases, such as bone related diseases, cancer, cardiovascular diseases, depression, multiple sclerosis, diabetes and obesity. It's also linked to modulation of inflammatory processes and testosterone production. However, except for bone related diseases where there is somewhat strong consensus, there is contradictory evidence between vitamin D levels and the rest of health issues. This might be due to 25OHD tests not being properly standardized and poor experiment design [D13, D34, D67]. 

 

Bone Health 

 

Vitamin D deficiency led to a lower amount of calcium and phosphate in blood, resulting in abnormal bone grow, rickets, brittle bone diseases, osteomalcia, or osteoporosis [D68]. 

 

Rickets is a rare disease in developed countries that increase significantly from 10% to 70% in places of Africa, Middle East, and Asia [D69] , and has higher prevalence in recent year than in the pass, up until the introduction of vitamin D and Calcium supplements in infant food [D69-71]. Most forms of rickets are due to vitamin D deficiency in children, while severe rickets cause developmental delay, hypocalcemic seizures, tetanic spasms, cardiomyopathy, and dental abnormalities [D70, D71]. Almost all patients with rickets had been partially or exclusively breastfed [D70, D72]. There are different definitions and diagnostic criteria to define osteomalcia, but prevalence is about 3.7% with higher rate in women [D73]. 

 

Osteoporosis is characterized by the deterioration of bone tissue leading to bone fractures. About 15% of people older than 50 and 70% older than 80 suffer osteoporosis worldwide [D74]. Osteoporosis is caused by the lack of calcium intake, while osteomalcia and rickets are caused by the lack of vitamin D intake; however, a lack of vitamin D intake contributes to the severity of osteoporosis [D9]. For the general older adult population, some studies show that vitamin D and calcium supplementation increase slightly bone density, display reduce fracture rates. [D9, D75-D77], while other studies show no difference [D77-D84] In the United States, bone density, mass, and fracture risk are correlated with 25OHD in white individuals and Mexicans but not for black individuals [D18, D85, D86] 

 

Muscle Weakness 

 

Vitamin D assists in the development of muscle fibers. A proper support of the bone structure is needed for optimal bone health, so indirectly, vitamin D also helps bone development not only via calcium absorption but also due muscle growth. Inadequate levels of vitamin D can lead to myopathy. Sadly, experiments with vitamin D supplements also include calcium supplements. As such, being able to differentiate between the advantages of calcium and the advantages of vitamin D independently is challenging. Studies are also not standardized in the amount of nutrient or the time in which they are administered; this is important as nutrient absorption varies with time of the day and with the food that you are eating. For example, calcium inhibits iron absorption, while vitamin C enhances iron absorption [D87, D88]. Another example is that proteins need carbohydrates to be absorbed, vitamins can be water or fat soluble, and each group require different food setup to maximize absorption; all of these variables are rarely taken into account during the intake of supplements in studies. 

 

For a general population, studies have shown either inconsistencies [D89] on the effects of vitamin D supplementation with respect the muscle strength and muscle decline, or no correlation [D90, D91]. 

 

Cancer 

 

Some studies suggest that vitamin D inhibits carcinogenesis and slows tumor growth. They also suggest that it has anti-inflammatory effect, can modulate the immune system, it triggers cells to proapoptotic, and is antiangiogenic [D9, D92]. 

 

Observational studies show correlation between low 25OHD and cancer mortality [D77, D93, D94]. Some clinical trials support the hypotheses of these observational studies [D95-97]. And a particular clinical trial supports that vitamin D supplementation in a general population delays cancer appearance by 5.3 years in median [D98]. Other suggest no correlation effect [D77, D99]. In both cases, there is a confusing wording in the effect of vitamin D or calcium supplementation, suggesting that individuals with low 25OHD have an increased risk of cancer mortality that is fixed by supplementation, while individuals with normal 25OHD levels have no increased risk and thus the supplementation is useless because no association is linked between supplementation and cancer mortality. This makes confusing the effect of supplementation and mortality because it is not the supplementation itself what reduce the cancer mortality but having appropriate 25OHD levels. However, we have already established that low levels are cured using supplementation provided that you don't suffer for any of the co-morbidities that prevent 25OHD absorption. 

 

In breast cancer there is contradictory evidence, from inverse levels 25OHD and mortality, to the opposite, passing through no correlation. [D100-D106]. Lung cancer shows no association between circulating concentrations of vitamin D and risk of lung cancer [D107]. Pancreatic cancer: No correlation and positive levels of 25OHD associated with higher cancer risk [D108-D110]. Colorectal cancer: Inverse levels 25OHD and mortality specially in women, no correlation, and positive correlation between vitamin D supplements and calcium supplements with the development of polyps [D100, D111-D113]. Prostate cancer: Contradictory evidence between 25OHD levels and cancer risk, mortality, and length [D114-D122]. 

 

Cardiovascular diseases 

 

In the context of cardiovascular diseases (CVD), vitamin D regulates the RAS [D123] (renin-angiotensin-system), which regulates blood pressure, systematic vascular resistance, electrolyte balance and fluid. Also regulates vascular cell growth, fibrotic pathway, and inflammatory pathways. Indirectly, vitamin D enhance the absorption of calcium and potassium, which are both elemental pieces in the cardiac action potential for both pacemaker cells and contractile cells, specially the later as the absolute refractory in contractile cells is essential to prevent tetanus. Vitamin D deficiency has been shown to be associated with vascular dysfunction, arterial stiffening, left ventricular hypertrophy and hyperlipidemia [D124]. 

 

As such, observational studies have linked vitamin D with lower CVD risk. A positive association between high 25OHD and lower CVD events (myocardial infarction, ischemic heart disease, heart failure, and stroke) and mortality [D125]. There is also a study that shows an association between high and low 25OHD levels and CVD [D126]. And other studies found correlation between low 25OHD and high CVD [D127, D128].  However, clinical trials contradict all of this [D77, D98, D129], with some studies suggesting that it protects against cardiac failure, but not against myocardial infarction or stroke [D130]. 

 

Supplements have also shown to reduce total cholesterol and LDL, but not HDL [D131], they have contradictory effect on blood pressure for normal weight patients [D132, D133], and when taken with calcium increase blood pressure in overweight and obese patients, while having low 25OHD also increase blood pressure [D132, D134]. 

 

Depression and other mental illnesses 

 

Vitamin D receptors are present in several areas of the brain, and it is believed to be involved with depression, dementia, schizophrenic-like disorders, hypoxic brain injury, and other mental illnesses; as well as neuronal development and a decrease of microglial inflammatory function. [D135-D139]. 

 

But once again, clinical trials found that the administration of vitamin D supplements were not linked with reduction of depressive symptoms [D140-D144]. It is worth mentioning that none of these studies tested the combination of vitamin D supplements, plus antidepressant, in individuals with low 25OHD. 

 

Multiple sclerosis 

 

Multiple sclerosis (MS) chronic disease of autoimmunity or oligodendrogliopathy origin [D145] where an inflammation of the cover of nerve cells in the brain and spinal cord disrupt the ability of the nervous system to transmit signals properly. As a result, it can present almost any neurological symptoms. MS occurs less frequently near the equator and more frequently near the poles [D146]. This led researchers to investigate whether sun exposure has anything to do with MS, which is of course related to vitamin D absorption [D77]. 

 

Studies have shown the presence of low 25OHD before and after MS begins [D146-D149]. Others show that normal 25OHD levels reduce risk of contracting MS and decrease the time in between relapses once it has started [D150]. But clinical trials once more contradict these findings [D146, D151]. 

 

Diabetes and glucose homeostasis 

 

 Vitamin D stimulates the secretion of insulin via vitamin D receptors on the pancreatic beta cells, which reduces insulin resistance, and it is also involved in the insulin signaling pathways [D152-D154]. Some studies have shown a relationship between vitamin D and risk of diabetes [D9], others have shown an inverse correlation between vitamin D and blood sugar [D155]. 

 

Clinical trials tell that there is no relationship. Including vitamin D not improving insulin sensitivity in overweight and obese patients [D156], supplements not having an effect in glucose homeostasis or insulin resistance [D77, D157], no prevention of going from pre-diabetes to diabetes [D158] , the same for overweight or obese with normal 25OHD levels, but yes for those with lower 25OHD levels [D152, D159], and no benefits on individuals that already have diabetes [D154]. 

 

Obesity 

 

As vitamin D is fat soluble, more fat deposits in the body will sequester more vitamin D. Observational studies suggest that weight has an inverse correlation with 25OHD levels [D41, D42, D160, D161], and even reduce weight gain in postmenopausal women [D162] but vitamin D supplementation doesn't help with weight loss [D163]. 



 
\begin{comment}


...instead of pushing for open source / free software. Free software is often less expensive or completely free than proprietary counterparts. Selling your soul to MS. Free software provides transparency and greater control over what the software is doing on your system; this issue alone should be enough to ban the use of any medical data with anything that is not open source product.O pen standards, making it easier to share data between different applications. Free software is based on the ethical principle that software should be freely available to everyone

Learning curve: Free software may have a steeper learning curve for users due to customization options and complexity.



There is no valid in place solution to share results between collaborators. Talk GIT.


Also, OneDrive doesn't support anonymous link sharing. For any external collaborator, or simply curious minded person who want to see the results of the analysis, you need to ask permission that need to be granted manually (by me), and in order to do so you need at least a mail account. But you need a Microsoft mail account if you want to access the full range of functionality. Lastly, due security issues, you also need a UiT account since we have a system wide ban on external collaborators sharing documents without an organization mail account.

OneDrives can comply with UiT's privacy of data policies, however, further configuration is required via Azure in order to setup a proper secure server for your project. Dropping red and black data in a common OneDrive folder is not allowed.


\subsubsection{TSD}

\href{https://www.uio.no/english/services/it/research/storage/filesender.html}{Services for sensitive data (TSD)} is a platform for collecting, storing, analyzing and sharing sensitive data in compliance with the Norwegian privacy regulation. TSD is used by researchers working at UiO and in other public research institutions (the UH-sector, universities, hospitals etc.). The TSD is primarily an IT-platform for research even if in some cases it is used for clinical research and commercial research.

TSD has the advantage that can be use a virtual remote machine, which is quite convenient if you don't want to be bothered with security issues related to data privacy. This include the possibility of several people working on the same documents (without version control software) and a common space to share results securely.

The main disadvantage of TSD is that getting something out of TSD is also a bureaucratic ordeal. For example, if you want to write an article, and you want to include a figure you generated in there, you might have to wait several weeks before you are able to extract the figure from the server; basically until somebody look at it manually and check that is not violating anyone privacy. And this will repeat each time you generate any new file. Even extracting your own code is time consuming task because you also need to check that the code don't contain privacy leaks.

Other minor inconveniences is that, at the time of writing this, R software was limited to Windows architecture only. You also need to pass periodically another bureaucratic layer to keep access to your project, as project expire over time.

TSD however has the potential to become the best cloud-sharing option, by far. It already has in place the physical infrastructure, with the hard drives with the actual data within Norwegian borders, and software that allow for the execution of a remote virtual machine which you can use anywhere you want. It just need to include it own private GIT system where you can store and importantly retrieve the parts of your project that are not red or black data.

The solution that the people in Sommarøy were telling about, comment here




%To convert all these original data, into a more human and computer friendly version, I use the script "dataCleaning.R". Once the data is "clean" and we filter out the values that we don't want, then we can proceed with our analysis.

%In total, among these 5 files, we have available 944 columns that are unique. Which we later on transform into about 656 variables, with that number increasing depending on whether you consider each individual network as independent datapoints or all part of the same dataset.


\subsection{Data redundancy}




\subsection{Second hand obesity}

Society plays an instrumental factor in how many opportunities an individual have to gain weight. Obesity is not a transmittable disease, and is not possible for a person to eat and make another person gain weight, and yet people can influence your obesity in several ways:

    \begin{description}
    
        \item[Peer pressure:] People often eat and drink more when they are with friends or in a social setting. Peer pressure can make them indulge in unhealthy food choices that can contribute to weight gain.

        \item[Marketing and advertising:] The food and beverage industry spends billions of dollars each year on marketing and advertising. This can influence people's eating habits and preferences, leading to the consumption of calorie-dense and nutritionally-poor food options.
        
        \item[Family and cultural norms:]  Cultural values and familial customs often shape an individual's eating behavior and dietary habits. Family members and cultural traditions may promote high-fat, high-sugar, and high-salt diets that can lead to weight gain.
        
        \item[Social norms and attitudes:] In some societies, obesity is viewed positively, and a person who is overweight may be seen as wealthy or healthy. This can lead to individuals feeling less motivated to lose, or control, their weight.
        
    \end{description}

Is the individual ultimate action of high energy intake to low energy output the responsible for obesity levels. 


\section{High influence (wording) }

\section{Limitations}

\subsection{Nutritional data sucks}

There are some conditions which makes obesity hard to manage are not influence by friends.  

Hypothyroidism lower metabolic rate decreasing energy output. It also impair kidney function leading to unbalance fluid filtration, but this edema is due water and salt, not fat.

Individuals with Prader-Willi syndrome have an uncontrolled and insatiable appetite, leading to excessive weight gain.

Conditions in which the adrenal gland is affected lead to an unbalance of hormones, in particular cortisol. Cortisol stimulate breakdown of adipose tissue, but also suppress ability to maintain muscle mass which leads to slower metabolism. This is the case of Cushing's syndrome or Congenital Adrenal Hyperplasia.

Leptin stimulate the brain regulating the feeling of hunger. The more leptin you have, the more full you feel and the less you eat. Leptin levels increase as fat mass increases and viceversa. But if the brain is constantly overstimulated by leptin you get leptin resistance, causing more hunger feeling leading to overeating. As obesity increases leptin, obesity lead to leptin resistance, leading to more obesity in a positive feedback loop. Conditions that affect leptin are Congenital leptin deficiency. Leptin increses with emotional stress, sleep apnea, estrogen, and obesity. Decreases with PA, and testosterone.

Any cause that leads to insulin resistance, such as Polycystic ovary syndrome (PCOS), Insulinoma, or type II diabetes.








\section{The Need for Up-to-Date Skills and Tools among Health Professionals}


\subsection{Programming languages}

\subsubsection{ LaTeX }

\LaTeX \cite{ref:latexintro} is a system for preparing written documents. The main characteristic is that, unlike in Microsoft Word, LibreOffice, and other similar software where you see the final result of what you are written live as you edit it, in LaTeX the user types in plain text keeping style and content separated, in a similar way as HTML+CSS works. Later on the code is compiled into a PDF document where you can see the final stylized result.\vspace{3 mm}

The main inconvenient of LaTeX is the learning curve, however LaTeX is widely used in all academia fields for the communication and publication of scientific documents. In my opinion, the inconvenience of having to learn LaTeX outweighs, by far, the amount of time that you are going to save editing documents in a "What You See Is What You Get"  \cite{ref:wysiwyg} editor. To the point that mathematics communications alone would be near impossible to perform without this software.\vspace{3 mm}

As opposite to Microsoft Word where you can synchronize automatically with OneDrive, Latex doesn't have by itself a collaborative interface and you are dependent of using a control software, such as GIT {\tiny [\ref{sec:git}]} , or using online services, such as Overleaf {\tiny [\ref{sec:overleaf}]}. In any case, this problem can also be adverted by setting automatic version control scripts; which to be fair, scare people outside mathematics, informatics, and physics fields.\vspace{3 mm}

\subsubsection{SPSS}

The letters SPSS \cite{ref:spss} stands for "Statistical Package for the Social Sciences". Is a proprietary statistics software developed for social sciences researchers who have very limited, or none, knowledge of programming.\vspace{3 mm}

The advantage of SPSS is that is composed of easy to use drop-down menus. If a person has some basic knowledge of statistics, SPSS is very easy to use even for complex multivariate analysis.\vspace{3 mm}

The shortcomings are that you can't do extensive scripting in SPSS, so anything that we do in this project, such as generating results automatically, generating latex code automatically, or generating websites automatically, won't be possible. During the course of the project we also tend to change biological definitions, such as, what is a teenager, or what does it mean to be a carrier of a bacteria. Running manually all the drop-down menus in SPSS, again and again, each time we decide to change the definition of a disease, or who is the target population to analyse, would be an insurmountable time consuming task. As such, is impossible to use SPSS in this project.\vspace{3 mm}

Other negative issues are the proprietary license cost, worsened by the fact that it has a software as service license, plus the issues with close software security. Adding to this, there are statistical method within several libraries available in R that aren't in SPSS, namely almost everything that has to do with network analysis. For fairness, mention that it has a trialware license where you can use the software for free, but I would strongly advice against wasting your time using it given all the limitations that presents.\vspace{3 mm}

Some of the Fit Future data is sadly only available in SPSS files. However there are plenty of libraries that overcome this so you can read the data outside SPSS without having to do lengthy transformations.\vspace{3 mm}

\subsubsection{R}

R don't have classes. The classes that R provide are to classes, the same that the Frankenstein's monster is to humans.

R \cite{ref:rproject} is a programming language specialized in statistical analysis. Is very popular among the data analysis community and praised everywhere. But not by me. R has a lot of shortcomings and limitation as a programming language, which I discuss in great detail at the end of this document.\vspace{3 mm}

In essence, R an evolution of the APL programming language (1966) \cite{ref:APLlanguage}, that later evolved into S (1976) \cite{ref:Slanguage}, which latest evolved into R (1993). Because of this R has evolved dragging the plenty of disadvantages of it predecessors, and is a nightmare for anyone who has learned to do object orienting programming (such as with C++). To the point that, even the most basic of operations which is the assignment operator, is nowadays a subject of debate as whether is appropriate to use "<-", "=", ":=", or "<<-",  \cite{ref:Rassignoperators}, depending on what environment you want your variable to be saved. This is an unnecessary complicated concept for anyone who understand the classic concept of scope of a variable, which it is a very simple concept used in any other programming language. It doesn't have a proper class structure, it doesn't have enumerations, it mixed access operators ("[]" for vectors, "[[]]" for lists), it doesn't have polymorphism, it doesn't have pointers or any type of direct memory access commands for optimization, you can't make function that distinguish between passing as argument, reference or constant reference, it doesn't have constants variables, it doesn't allow for variable type declaration, boolean logic involving NA values defy previously well established programming patters (R define FALSE and NA = NA ; which is not true, FALSE and anything has always been FALSE since George Bool was born), it doesn't have switch structure, and many more.\vspace{3 mm}

R is incredibly inefficient, to the point that you need to use the Rcpp \cite{ref:rcpp} library and throw your data into a C++ program that runs your heavy operations with a big dataset if you want your code to work in an acceptable amount of time. But none of this matter in comparison with the amount of bugs and difficulty debugging your code that R has in comparison with other languages. And that is what makes, in my opinion, the language inappropriate for scientific research.  The general population are not well trained to proper testing functions; not even IT engineers do proper testing! and as such, there must be hidden bugs everywhere nowadays that influence the outcome of studies.\vspace{3 mm}

The only advantage of R is that is easy to learn and good for tiny projects, which makes it very popular. But just to remind the reader, Java has been the most popular programming language in the last 20 years according to the TIOBE index \cite{ref:tiobeweb} \cite{ref:tiobewiki}, and is another good example that popularity doesn't correlate with quality.\vspace{3 mm}

Another advantage is that R has plenty of libraries for statistical analysis. This is not a competitive advantage anymore as Python is easier to learn, has already all libraries that you would ever need, but doesn't have as many problem and disadvantages that R has.\vspace{3 mm}

As a personal advice, if you need a general purpose programming language, learn to program with C++, and if that is too complicated for you, use Python instead. And if you want a language that is very specialized in mathematics use MATLAB. And if you need something that is even more niche, and you need one particular area of mathematics use PROLOG for example. But don't go into the R rabbit-hole (much more less SPSS of course).\vspace{3 mm}

\subsubsection{C/C++}

While C++ would be my preferred program of choice, it is not the most popular language for data analysis, in part for the initial learning curve that has over R or Python. There is some work done in C++ though, in particular generating websites.\vspace{3 mm}







%*****************************************
\section{Social influence on infection diseases}
%*****************************************


The development of a lens with enough power to observe microbes was pioneered by Antonie van Leeuwenhoek (1632-1723), while Louis Pasteur (1822-1895) and Robert Koch (1843-1910) made significant contributions to our comprehension of the hidden realm of microbes. However, humanity has known early notions of diseases since ancients civilization and that they were somehow transmitted around the air, and getting close to a sick person was a bad idea and a likely way of getting infected.

Infectious diseases can be transmitted through social interactions in 3 main ways:

    \begin{description}
    
        \item[Direct physical contact:] When people come into direct physical contact with one another, infectious diseases can be transferred through skin-to-skin contact, saliva, blood, or other bodily fluids.
        
        \item[Indirect physical contact:] Infectious diseases can also be transmitted when people come into contact with objects or surfaces that have been contaminated with the pathogen and then touch their mouth, nose, or eyes.
        
        \item[Airborne transmission:] Infectious diseases can be spread through coughing, sneezing, and talking, as respiratory droplets can contain the pathogen and infect others who inhale them.
        
    \end{description}

It is no surprise to find that \gls{staph} or any other pathogen is transmissible via social influence in our population. However our contribution was to be able to put numbers to it and being able to tell how much.



%*****************************************
\section{Friends are the sunshine of life}
%*****************************************

Similar to obesity, people tend to do similar activities of those of their friends. You might be sitting on front of the PC playing videogames all day long by yourself or with other friends and never see the sunlight, or you can become a human lobster by suddenly going in a vacation in the middle of July to Costa del Sol with your family. Both cases are unhealthy extremes examples, but nevertheless illustrates how your behaviour influence UVB exposure and vitamin D synthesis.

Furthermore, we already discussed how society can dictate your nutritional habits, which may or may not be rich on D2+D3.


Contraceptives


But once again, it falls on the patient responsibility to guard himself from the Sun, and eat properly.




\section{The data cleaning process}



\section{All programming languages lack a higher analysis abstraction}



\section{Automatic analysis}




%*****************************************
\section{Interdisciplinary field}
%*****************************************

My background is computer science. As such I started having little to no idea of immunology, microbiology, endocrinology, and so for. And my knowledge on statistics and graph theory was probably lower that it should has been. Working and learning from so many topics at the same time has been a challenging work.

Furthermore, the need for many collaborators from such fields will arise sooner or later. And having to find a time that suits everyone involved has been proven to be an almost insurmountable task of nobody's fault other than the lack for humans to be able to work for more than 24h a day. This leads to progress being achieve at a very slow pace in each of the topics individually. For example, we had the biomarkers and obesity results on stand-by for about 1 year and half, and the vitamin D results for almost a year. My recommendation for future researches in a similar position would be, try to work at everything at the same time and overlap waiting times as much as possible.



% https://www.scribbr.co.uk/thesis-dissertation/discussion/

\begin{comment}
To improve the discussion chapter, consider the following:

Identify common themes: Look at the discussions in each paper and identify common themes or findings that are relevant across multiple papers. This will help you identify the big picture implications of your research and tie the individual papers together.

Offer a synthesis: The discussion chapter should offer a synthesis of your research findings. This means that you should connect the dots between different papers and draw conclusions about what your research means as a whole.

Provide a comprehensive analysis: Consider expanding beyond the direct implications of your research and providing a comprehensive analysis of the field or discipline in which your research is situated. This can provide context for your research findings and help readers understand the broader implications of your work.

Address limitations and future directions: In addition to discussing your findings, it is important to address the limitations of your research and provide suggestions for future directions for research in your field.

Consider the audience: Keep in mind the audience for your discussion chapter. It should be accessible to a broad range of readers and written in clear, jargon-free language.

Overall, the discussion chapter should tie the individual papers together and provide a comprehensive analysis of your research findings. It should offer a synthesis of key themes and draw conclusions about the significance of your research within the broader context of your field or discipline.
\end{comment}



\section{Challenges in obtaining and utilizing data}

\subsection{Introduction}

The UiT is striving to continuously improve its scientific practices. During my work in this Ph.D., I have examined certain areas that require significant attention. Specifically, the institution's data management practices could be improved to ensure greater transparency, accuracy, and accessibility. This is a matter of great concern as it not only will improve scientific integrity, but also help to save very valuable time in future research which will allow the institution to help the public more efficiently.

During this section, I am optimistic that highlighting these challenges will lead to a greater emphasis on improving data management practices, and the institution will continue to make significant contributions to scientific research.


\subsection{Legal}

%\subsubsection{Privacy levels}

In our institution data is classified according to the level of privacy required for its handling  \cite{ref:DataColors}. There are 4 privacy levels:

\begin{itemize}

	\item \colorbox{ForestGreen}{\textcolor{white}{\textbf{GREEN}}} Open and freely available for everybody. Here we include the results of our statistical analyses so long they don't contain enough information to identify a person.
	
	\item \colorbox{Yellow}{\textcolor{black}{\textbf{YELLOW}}} Restricted data. The kind of data that is not dangerous to be leaked, but at the same time is reasonable and is only handled by a selected amount of individuals. After the further anonymization of the data, and cleaning entries in which personal IDs were included in the data, we do not have any yellow data.
	
	\item \colorbox{Red}{\textcolor{white}{\textbf{RED}}} Confidential data. Here is the data that is protected by law or other regulations. In our case, there is information about the personal health of individuals, who even though are anonymous, might get identified with the proper set of data points. All our original data is considered to be red data or black data.
	
	\item \colorbox{Black}{\textcolor{white}{\textbf{BLACK}}} Strictly confidential data. In the official Norwegian instructions for information protection, black data is used if the university, its partners, the public interest, or individuals, may be subject to considerable harm if the information is exposed to third parties. All the original data collected for Fit Future is under this category. Later on, the data gets anonymized and shared around under red conditions.
	
\end{itemize}


\subsection{Original files}

The data used in the analysis comes from several different original files. In table \ref{table:Original_Files} we describe the origins of each file and a brief description of what it contains. Some of the variables are repeated across these files and we will discuss the issue of data redundancy across Fit Future later on.

    \begin{table}[H]
    
        \caption{Summary of the available Fit future data and received date.}
	\label{table:Original_Files}
        
        
	    \renewcommand{\arraystretch}{1.7}
            \scalebox{0.65}{
            \centering
    
	    \begin{tabular}{| l | p{10cm} l | l}
    	    \hline
        	\rowcolor[HTML]{FFAAAA}

	        \textbf{Filename} & \textbf{Description} & \textbf{Date received} \\ 
    	    \hline 

        	\multicolumn{1}{l|}{\detokenize{  PERSKEY_Rafael_s aureus_19022020.xls }} & For the FF1 period only, all the \textit{S. Aureus} information regarding direct cultures and enrichment broths, SPA types and dates of cultures. All the social network information including network representativeness grading. Some phenotype variables: ID, sex, age, high-school information, smoking habit, snuff habits, sports habits, BMI, use of antibiotics including frequency and brand & 2020/02/21\\ 
        
	        \multicolumn{1}{l|}{\detokenize{data_ut_11Juni2019.dta}} & For the FF1 period: Anthropometry data, diseases, some medication usage, menstruation cycle, hormonal contraceptive information, some of the blood serum analysis variables, and puberty development. For the FF11 and FF12 periods, the follow-up status on colonization. & 2021/02/26 \\
         
    	    \multicolumn{1}{l|}{\detokenize{eutro_rafael_paakoblet.sav}} & For the FF1 period only. Full medication data, full blood serum, full biomarkers, household information, ethnicity, hygiene, and sunbathing & 2021/08/04 \\ 
        
        	\multicolumn{1}{l|}{\detokenize{eutro.sav}} & For FF1 period, diet information. For the FF12 period, follow up on the social network information.  & 2021/10/05  \\ 

	        \multicolumn{1}{l|}{\detokenize{Perskey_FF2 antropometri.xlsx}} & For the FF2 period, basic anthropometry variables (no DEXA scans available so far for any period).  & 2022/05/10  \\ 
            
    	\end{tabular}
            }
	\end{table}

For reference, I started my Ph.D. on 2019/06/02. Notice that I don't get the social network data until 8 months later, and the complete data needed for Paper A until, in total, 20 months have passed. Access to RED and BLACK data is heavily restricted, and going through the process of asking for new data, justifying the request, waiting for approval, and finally getting the data is a very long-term process that should not be as cumbersome. There are other issues, such as no biomarkers data until 2 years into the Ph.D. in order to work on Result II (and other unmentioned papers), and any FF2 data arrived until 3 years into the Ph.D. with only a short obesity-related data with no social network information. So far I have also not gotten access to the DEXA data which is critical to analyzing proper bone mineralization health which is very related to vitamin D, PA impact, and real fat percentage levels not just BMI.

Luckily I have put a lot of effort into the data cleaning process and developing better R practices, which will be discussed later, which have led to being able to generate results very quickly, but the writing process has been bottle-necked due to the delay in data and as a result, most of the papers related to this thesis are in a limbo state of having results worth publish but still not fully finished. We even had a situation in which, while working on Paper A, the other main author has permission to have the data, but I didn't. This ran on for about 3 months when I had to prepare scripts blindly in advance for when I received the actual data and compiled all the results quickly.


\subsection{Not much open access and no open source}

From the UiT in particular and the Norwegian government in general, there's a lot of emphasis on publishing articles that are free and fully available to the public. This is a great initiative that must be enforced in order to strive for better research practices, and its benefits have been documented thoroughly. However, these efforts fell short in several aspects.

In science, you haven't discovered something until you discover it and somebody else reproduces it. The concept of reproducibility has become increasingly important in academics during the last few years as a form of adding weight to your academic papers. All our code is released under Affero GPL3.0 license \cite{ref:afferoGPL}, and as such, all methods are publicly available for anyone to share, modify, or distribute. However, we can't say the same for our data, as it falls under the red category and is not openly available to share it around. Our reproducibility is restricted from the general public, and you will need to ask for the proper data permission if anyone wants to reproduce the results.

All the original data is stored in proprietary formats, whether it is ".xls", ".dta", ".xlsx", or .sav". All of these need to be converted into ".csv" formats with exactly the same information, which is a better standard for the computers. This is something that I shouldn't have done. This is something that should have been done 11 years ago when the data was gathered and saved for everyone to use. All of these files were converted into a more user-friendly CSV format without any modification to the data itself. We build upon these CSV files, later on, to apply all the needed transformations in the data-cleaning process explained in the method. However, all these efforts will fall short as it is impossible to merge all the cleaned data, and all the openness of it into the general data repository. All the data files need to be deleted from local computers after the Ph.D. is finished.

Another comment is that no open-source methodology is being promoted, instead the university is heavily biased toward using Microsoft products. This contradicts the hardcore rules regarding privacy as all single meeting hosted on Teams is transcribed and sent to foreign servers outside Norway which directly contradict the security policies trying to be implemented on the data. Also, despite its popularity, tools such as Word have limitations in terms of advanced features, formatting options, and graphics; sometimes struggle to maintain consistent formatting across different devices or when exporting or importing files. Word documents may not always be compatible with other open tools, documents, and platforms, which can cause issues when collaborating or sharing files.  Is my healthful advice to all professionals working in the biological and medical field within this university to ditch the use of Word online and learn how to work with Latex and GIT instead.

% Person reading these comments, I have PTSD from working on merging Word files!

Following in the close software, people in health sciences also tend to be biased towards using SPSS (Statistical Package for the Social Sciences), which is a proprietary statistics software developed for social sciences researchers who have very limited, or no, knowledge of programming. The advantage of SPSS is that is composed of easy-to-use drop-down menus. If a person has some basic knowledge of statistics, SPSS is very easy to use even for complex multivariate analysis. However, a significant shortcoming is the proprietary license cost, worsened by the fact that it has a "software as service" license, plus the issues with close software security. Adding to this, there are statistical methods within several libraries available in R or Python that aren't in SPSS, namely almost everything that has to do with network analysis and machine learning models. For fairness, mention that it has a trialware license where you can use the software for free, but I would strongly advise against wasting your time using it given all the limitations that presents. Another very strong negative point is that you can't do extensive scripting in SPSS, so anything that we do in this project, such as generating results automatically, generating latex code automatically, or generating websites automatically, won't be possible at all.


%---------------------------------------------------------------------------
% NORWEGIAN ABSTRACT (250-ish Words)
%---------------------------------------------------------------------------
\chapter*{Norsk sammendrag}
\addcontentsline{toc}{chapter}{Norsk sammendrag}

\textbf{Forskningsspørsmål:} Hovedmålet med denne doktorgradsavhandlingen er en utforskende undersøkelse av den sosiale nettverksdynamikken i åtte videregående skoler i Tromsø og i hvilken grad denne dynamikken bidrar til den generelle helsen og trivselen til studentene. Slik som i sammenheng med spredning av smittsomme sykdommer og overføring av negative helseeffekter. Videre har studien som mål å sammenligne observerte effekter på både sosiale og ikke-sosiale vertsfaktorer som sport eller rekreasjons rusmiddelfrekvenser.

\vspace{0.90\baselineskip}

\textbf{Metodologi:} Ved å bruke Fit Futures innsamlede data om vennskap, bruker vi simuleringer, homofili, $X^2$-tabeller, logistisk regresjon og tilfeldige skoger som hovedmetodene som brukes til å analysere sosial innflytelse i de tidligere nevnte emnene.

\vspace{0.90\baselineskip}

\textbf{Resultater:} Vi fant at det sosiale nettverket påvirker spredningen av \gls{staph} betydelig. Studenter i nettverket har en tendens til å ha lignende inflammatoriske biomarkører, \gls{25ohd}- og \gls{bmi}-nivåer. Noen videregående skoler har en tendens til å konsumere lignende nivåer av reseptfrie medisiner og har en tendens til å dele samme merke med reseptbelagte medisiner. Det er også en skjevhet når det gjelder bruk av rusmidler på videregående skole.

\vspace{0.90\baselineskip}

\textbf{Konklusjoner:} Sosial påvirkning er vist å være en betydelig komponent for alle disse variablene.

\vspace{0.90\baselineskip}

\textbf{Originalitet:} Bruk av ikke-parametrisk simulering og maskinlæringsmetoder for å estimere sosial innflytelse. Måler også sosial påvirkning på \gls{25ohd} og i en komplett inflammatorisk proteomisk analyse.

\vspace{0.90\baselineskip}

\textbf{Betydning:} Sosial påvirkning, enten det kommer fra virtuelle venner eller fysiske, er et økende interesseområde på mange felt. Spesielt innen epidemiologi så vi en økning i popularitet etter Sars-Cov-2-pandemien. Funnene i denne studien gir verdifull innsikt som kan brukes til å veilede utviklingen av fremtidige intervensjoner og offentlige politiske initiativer rettet mot å forbedre helse og velvære til elever i videregående skolemiljøer, og fungerer som en mal for studier som bruker andre populasjoner.

\vspace{0.90\baselineskip}

\textbf{Søkeord:} Sosiale nettverk, statistikk, epidemiologi, vitamin D, fedme, betennelser, tilfeldige skoger, resepter, legemidler.



\section{Ethics of social network analysis}

The Cambridge Analytica scandal is a controversy that happened in early 2018 involving the unauthorized harvesting of the personal data of millions of Facebook users. Cambridge Analytica, a political consulting firm, used a third-party app to extract data from Facebook profiles and connections without the consent of the users, which was then used to build psychological profiles of individual users for political advertising purposes. The data obtained by Cambridge Analytica was said to have been used to influence the 2016 US Presidential Elections and the Brexit referendum in the UK. The scandal damaged Facebook's reputation and resulted in multiple legal inquiries and calls for greater regulations of the tech industry and data privacy.

This led later on in time to the antitrust hearings of 2020, where the CEOs of Apple, Google, Facebook, and Amazon testified before the US Congress. The hearing focused on monopolistic behaviors, anti-competitive practices, and privacy concerns of tech giants. The CEOs denied any wrongdoing and defended their companies' roles in fostering innovation and competition in the tech industry. However, lawmakers expressed concern about the immense power and influence of these companies and questioned whether they needed to be broken up to ensure fair competition and protect consumers' rights and privacy. These are the same companies that developed COVID-19 trackers which led to further concern regarding patients' privacy. Another example of companies studying network connections is LinkedIn, a social platform for professionals. The platform analyzes job connections and makes predictions, recommendations, and insights based on the individuals' connections which have been criticized for the use of customer data without their explicit consent. Furthermore, these types of problems are not solved by simply not using the site's services. In February 2019, the German government ruled that Facebook is not allowed to obtain data from non-Facebook users without their previous consent; and this took a three-year investigation.

This leads to the question on whether how much these companies are able to gather data about our social contacts without our consent, and if this data can be used against our interests. Social Network Analysis makes this type of unethical approach easy and is something that should be brought forward to the public attention. The second question is whether there's a type of defense that users can use against unwanted influence by third-party actors. In our case, we care more about health issues which is one of the concepts that we try to highlight in this thesis, so users can be aware of how malignant influence can affect their health in a negative way. Whether the users follow our advice or not is left up to them.

The third question also highlights another limitation of social network analysis, which would be an ethical way for the medical community to collect social network data in a way that doesn't take forever to gather. The easiest way today is just to roll out apps similar to the COVID-19 trackers or directly get Facebook data. But as explained above, this is already a shady method and we don't want companies to have our data to use as they want. The process of this data gathering via classical ways such as questionnaires with the patients signing a disclosure agreement is considered ethically acceptable but takes too much time, which would defeat the advantages of early detection that this type of analysis offers.

% https://www.theguardian.com/news/2018/mar/17/cambridge-analytica-facebook-influence-us-election
% https://www.nytimes.com/2018/03/17/us/politics/cambridge-analytica-trump-campaign.html



\begin{table}[H]

    \caption{All translations applied from the "Other" + "Comment" variables.}

	\tiny

	\centering

	\renewcommand{\arraystretch}{1.5}

    \scalebox{0.85}{

	\begin{tabular}{|ll|ll|ll|}
		\hline

\rowcolor[HTML]{FF9999} 
{\color[HTML]{000000} Original}                 & {\color[HTML]{000000} Translation} & {\color[HTML]{000000} Original}           & {\color[HTML]{000000} Translation} & {\color[HTML]{000000} Original} & {\color[HTML]{000000} Translation} \\ \hline
afgansk                                         & Afghan                             & gambisk                                   & Gambian                            & norsk/kvensk                    & Norwegian-Kven                     \\
afghans                                         & Afghan                             & halvt afrikansk. faren min er fra afrika. & African                            & norsk/somalisk                  & Somalian                           \\
afghansk                                        & Afghan                             & halvt dansk                               & Danish                             & norsk/svensk                    & Norwegian-Swedish                  \\
africansk                                       & African                            & halvt italiensk.                          & Italian                            & norsktamiler                    & Other                              \\
afrikansk                                       & African                            & halvt norsk og halvt ghanesisk            & Norwegian-Ghanaian                 & palestinsk                      & Palestinian                        \\
annet, spesifiser her (1)                       &                                    & halvt tysk                                & German                             & pappa kommer fra belgia         & Belgian                            \\
awesome (1)                                     &                                    & islandsk                                  & Icelandic                          & polsk                           & Polish                             \\
belgisk                                         & Belgian                            & italiensk                                 & Italian                            & portugisisk                     & Portuguese                         \\
bodd over alt i verden                          & Other                              & kinesisk                                  & Chinese                            & russisk                         & Russian                            \\
brasiliansk                                     & Brasilian                          & min mor er finsk og min far en norsk.     & Finnish                            & russisk/norsk                   & Norwegian-Russian                  \\
bulgarsk                                        & Bulgarian                          & nederlandsk                               & Dutch                              & somalisk                        & Somalian                           \\
canadisk                                        & Canadian                           & nederlandsk og svensk                     & Swedish-Dutch                      & spansk                          & Spanish                            \\
colombiansk                                     & Colombian                          & nord-norsk lenangsværing                  & Norwegian                          & svensk                          & Swedish                            \\
dansk                                           & Danish                             & nord-norsk!                               & Norwegian                          & tamil                           & Tamil                              \\
eritrea                                         & Eritrean                           & nordlænning (3)                           &                                    & tamilsk                         & Tamil                              \\
eritreisk                                       & Eritrean                           & nordnorsk (3)                             &                                    & thai                            & Thai                               \\
fjellfinn (2)                                   & Sami                               & norsk og filipinsk                        & Norwegian-Philipine                & tyrkisk                         & Turkish                           \\
fordi jeg ikke er etniske norsk eller de andre. & Other                              & norsk somalisk                            & Norwegian-Somalian                 & tysk                            & German                             \\ \hline
	\end{tabular}

    }

    \label{table:Table_Etnicities}

\end{table}



\begin{table}[H]

    \caption{Table with the up to 5 self-reported chronic diseases asked in a subsection of the questionnaire.}

	\tiny

	\centering

    \label{table:Table_Common_Diseases_2}
    
	\renewcommand{\arraystretch}{1.5}

	\begin{tabular}{|lll|}
		\hline
		\rowcolor[HTML]{FFAAAA} 
		Questionary                                             & Medical                        & ICD10  \\ \hline
		
		Høysnuve (allergisk rinitt)                             & Allergic rhinitis, unspecified & J30.9                        \\
		ADHD (hyperkinetisk forstyrrelse)                       & ADHD & F90.9                            \\
		Cøliaki/glutenintoleranse                               & Celiac disease & K90.0                  \\
		Eksem                                                   & Eczema & L30.9                          \\
		Matvareallergi                                          & Food allergy & T78.4                    \\
		Migrene                                                 & Migraine, unspecified & G43.909         \\
		Laktoseintoleranse                                      & Lactose intolerance & E73.9             \\
		Depressiv lidelse                                       & Depression & F32.9                      \\
		Blodmangel/anemi                                        & Anemia, unspecified & D64.9             \\
		Søvnlidelse (innsovningsvansker o/el tidlig oppvåkning) & Imnsonia & F51.9                        \\	
		Diabetes mellitus type 1                                & Diabetes Type 1 & E10                   \\
		Angst                                                   & Anxiety & F41.9                         \\
		Spenningshodepine                                       & Tension headache (TTH) & G44.2          \\
		Gastritt                                                & Gastritis & K29                         \\
		Uspesifisert artritt                                    & Artritis & M13.4                        \\
		Hypotyreose                                             & Hypothyroidism & E03                    \\
		Astma                                                   & Asthma & J45.9                          \\
		Spiseforstyrrelse                                       & Eating Disorder & F50.9                 \\ \hline
				
		
	\end{tabular}
	
\end{table}



%In the following table, we show the original comment, and the disease plus ICD10 that we assigned. Notice that several people report more than one disease at once, in these cases the comment is duplicated in the table, but it is registered as two independent diseases. Also, the same disease as before could appear duplicated here as well. We again keep the record that is more specific:

\begin{table}[H]

    \caption{Diseases "Other" (1/3).}	

	\tiny

	\centering

    \label{table:Table_Other_Diseases}
    
	\renewcommand{\arraystretch}{1.5}

    \scalebox{0.85}{

	\begin{tabular}{|lll|}
		\hline
		\rowcolor[HTML]{FFAAAA} 
		Questionary                                             & Medical                        & ICD10  \\ \hline

        \rowcolor[HTML]{FFD1AA}        
		\multicolumn{3}{|l|}{Brain and general nervous system}   \\
		\hline   		
		
		Epilepsi-juvenil                                    & Juvenile myoclonic epilepsy, intractable, without status epilepticus & G40.B19 \\		
		epilepsi                                            & Epilepsy, unspecified, not intractable, without status epilepticus   & G40.909 \\
		Epilepsi                                            & Epilepsy, unspecified, not intractable, without status epilepticus   & G40.909 \\
        epilepsi fra 6 år og daglige magesmerter fra 10 år. & Epilepsy, unspecified, not intractable, without status epilepticus   & G40.909 \\        
        Epilepsi og Blødersykdom Won Willibr                & Epilepsy, unspecified, not intractable, without status epilepticus   & G40.909 \\		        
        Epilepsi og Blødersykdom Won Willibr                & Von Willebrand's disease                                             & D68.0   \\
        epilepsi fra 6 år og daglige magesmerter fra 10 år. & Unspecified abdominal pain                                           & R10.9   \\        
		Prob med finmotorikken + utredet                    & Disorder of central nervous system, unspecified                      & G96.9   \\
        Charcot-Marie-Tooth type 1A                         & Hereditary motor and sensory neuropathy                              & G60.0   \\
		cerebral parese                                     & Cerebral palsy, unspecified                                          & G80.9   \\
		Migraine                                            & Migraine, unspecified                                                & G43.909 \\
		en del hodepine                                     & Headache															   & R51     \\
		
		\hline

        \rowcolor[HTML]{FFD1AA}        
		\multicolumn{3}{|l|}{Psychological / Psychiatric}   \\
		\hline   		

		løpsk tale                     & Cluttering                            & F98.6 \\		
		Selvskading. Kutting.          & Intentional self-harm by sharp object & X78 \\
		Diagnostisert med ME (CFS)     & Chronic Fatigue Syndrome              & R53.82 \\
        Asbergers                      & Asperger's syndrome                   & F84.5 \\        
        Asberger syndrom               & Asperger's syndrome                   & F84.5 \\                
        atypisk autisme (diagnose)     & Autistic disorder                     & F84.0 \\		        
        Psykose                        & Unspecified psychosis                 & F29   \\ 
        Er deperimert fra 18 års alder & Other recurrent depressive disorders  & F33.8  \\ 

		\hline

        \rowcolor[HTML]{FFD1AA}        
		\multicolumn{3}{|l|}{Heart and blood}   \\
		\hline   		
        
        Mangler aortaklaffen, plages med tungpust og smerter i brystet. Medfødt.          & Other congenital malformations of aorta & Q25.4 \\		
		Født med AV-blokk. Innlagt pacemaker.                                             & Atrioventricular block, complete        & I44.2 \\
		artrius lesoria? ( blodåre som går rundt spiserør/ pusterør)                      & Other specified congenital malformation of peripheral vascular system & Q27.8 \\
        Bærer av blodsykdom. Mangler Von Willebrandt faktor.                              & Von Willebrand's disease                & D68.0 \\        
        tar jerntilskudd etter at prøve viser lave jernlagre (3-4 måneder siden diagnose) & Anemia, unspecified                     & D64.9 \\		        
        
		\hline        

        \rowcolor[HTML]{FFD1AA}        
		\multicolumn{3}{|l|}{Respiratory system}   \\
		\hline   		

        \rowcolor[HTML]{88CC88}        
		\multicolumn{3}{|l|}{Asthma cases}   \\
		\hline   		
        
		Sportsastma?                                                             & Asthma & J45 \\		
    	Anstrengelsesutløst astmasymptomer etter mycoplasmainfeksjon 2 år siden  & Asthma & J45 \\				

		\hline
        
        \rowcolor[HTML]{88CC88}        
		\multicolumn{3}{|l|}{Allergies to plants and animals}   \\
		\hline   		
         
		en del allergi, mest mot dyr        & Allergic rhinitis due to animal hair or dander         & J30.81 \\		
		allergisk mot pelsdyr               & Allergic rhinitis due to animal hair or dander         & J30.81 \\				
		Allergisk mot pelsdyr               & Allergic rhinitis due to animal hair or dander         & J30.81 \\		
		Allergi mot pelsdyr                 & Allergic rhinitis due to animal hair or dander         & J30.81 \\		
		allergi hund og kattt og midd       & Allergic rhinitis due to animal hair or dander         & J30.81 \\		
		allergi mot katt                    & Allergic rhinitis due to animal hair or dander         & J30.81 \\		
		allergi mot katt og hund            & Allergic rhinitis due to animal hair or dander         & J30.81 \\		
		Allergisk husdyr                    & Allergic rhinitis due to animal hair or dander         & J30.81 \\								
		Allergisk mot hund og katt          & Allergic rhinitis due to animal hair or dander         & J30.81 \\	
		allergisk mot hund, katt og timotei & Allergic rhinitis due to animal hair or dander         & J30.81 \\								
		allergisk mot hund, katt og timotei & Allergic contact dermatitis due to plants, except food & L23.7 \\								
		allergi mot pelsdyr og timotei      & Allergic rhinitis due to animal hair or dander         & J30.81 \\								
		allergi mot pelsdyr og timotei      & Allergic contact dermatitis due to plants, except food & L23.7 \\						
		insektsallergi                      & Other insect allergy status                            & Z91.038 \\	
		
		\hline		

        \rowcolor[HTML]{88CC88}        
		\multicolumn{3}{|l|}{Allergies to pollen and others}   \\
		\hline   		
        
		allergi for pollen, pelsdyr og melkeproteiner og noe hun ikke husker. & Allergic rhinitis due to animal hair or dander & J30.81 \\				
		allergi for pollen, pelsdyr og melkeproteiner og noe hun ikke husker. & Allergic rhinitis due to pollen                & J30.1 \\			        
		pollenallergi og allergi mot gulrot og epler, hund katt og hest.      & Allergic rhinitis due to animal hair or dander & J30.81 \\				
        pollenallergi og allergi mot gulrot og epler, hund katt og hest.      & Allergic rhinitis due to pollen                & J30.1 \\			        
		Pollenallergi                                                         & Allergic rhinitis due to pollen                & J30.1 \\		
		pollenallergi hver sommer                                             & Allergic rhinitis due to pollen                & J30.1 \\			        
		Allergi                                                               & Allergic rhinitis, unspecified                 & J30.9 \\			        

		\hline

	\end{tabular}
 
	}
 
    
	
\end{table}


\begin{table}[H]

    \caption{Diseases "Other" (2/3).}	

	\tiny

	\centering

    \label{table:Table_Other_Diseases_2}
    
	\renewcommand{\arraystretch}{1.5}

    \scalebox{0.85}{

	\begin{tabular}{|lll|}
		\hline
		\rowcolor[HTML]{FFAAAA} 
		Questionary                                             & Medical                        & ICD10  \\ \hline

        \rowcolor[HTML]{FFD1AA}        
		\multicolumn{3}{|l|}{Bones diseases and Physical traumas}   \\
		\hline   		
		

        \rowcolor[HTML]{88CC88}        
		\multicolumn{3}{|l|}{Other disorders}   \\
		\hline   		

		hypermobile ledd                                    & Hypermobility syndrome      & M35.7 \\		
		senebetennelse i venstre legg siste to år. Tørr hud & Calcific tendinitis         & M65.20 \\
		senebetennelse i venstre legg siste to år. Tørr hud & Xerosis cutis               & L85.3 \\				
		Barneleddgikt                                       & Juvenile arthritis          & M08 \\
		Reumatisme. Juvenil ideopatisk artritt. $^1$         & Juvenile arthritis          & M08 \\		
		hypermobile ledd og mulig bechtrew, leddgikt $^2$    & Other arthritis             & M13 \\				
        Dystrofia Myotonica                                 & Myotonic muscular dystrophy & G71.11 \\        

		\hline

        \rowcolor[HTML]{88CC88}        
		\multicolumn{3}{|l|}{Head and neck}   \\
		\hline   		
		
		Myalgisk encfalopati                              & Myalgia     & M79.1 \\
		Nakkeplager. Fått fysikalsk behandling. I bedring & Cervicalgia & M54.2 \\
		nakkesmerter                                      & Cervicalgia & M54.2 \\
		Nakkesmerter på høyre side etter skade            & Cervicalgia & M54.2 \\
        
		\hline
		
        \rowcolor[HTML]{88CC88}        
		\multicolumn{3}{|l|}{Spine and back}   \\
		\hline   		

		Skoliose                                                                             & Scoliosis                                            & M41 \\
		skoliose                                                                             & Scoliosis                                            & M41 \\
		Scoliose. Brukt korsett i 1 år 2009.Har ryggsmerter.                                 & Scoliosis                                            & M41 \\
		Scoliose. Brukt korsett i 1 år 2009.Har ryggsmerter.                                 & Dorsalgia, unspecified                               & M54.9 \\		
		kne,rygg og skulderskade etter en ulykke.  Skoleriose i rygg                         & Scoliosis                                            & M41 \\		
		kne,rygg og skulderskade etter en ulykke.  Skoleriose i rygg                         & Dorsalgia, unspecified                               & M54.9 \\							
		kne,rygg og skulderskade etter en ulykke.  Skoleriose i rygg                         & Unspecified superficial injury of knee               & S80.91 \\				
		kne,rygg og skulderskade etter en ulykke.  Skoleriose i rygg                         & Unspecified superficial injury of shoulder           & S40.91 \\			
		Beinmargsødem. Prolaps i korsryggen.                                                 & Other specified intervertebral disc displacement     & M51.2 \\
		Bechtrew                                                                             & Ankylosing spondylitis of unspecified sites in spine & M45.9 \\
		Kroniske ryggsmerter etter en idrettskade. Går nå med nakkekrage etter en nakkeskade & Dorsalgia, unspecified                               & M54.9 \\	
		Smerter i korsryggen i minst 4 år                                                    & Dorsalgia, unspecified                               & M54.9 \\
		Plaget med vondt i ryggen i flere år                                                 & Dorsalgia, unspecified                               & M54.9 \\
		Plages med ryggen siden den ene foten er lengre enn den andre $^3$                    & Dorsalgia, unspecified                               & M54.9 \\

		\hline
		
        \rowcolor[HTML]{88CC88}        
		\multicolumn{3}{|l|}{Hips and legs}   \\
		\hline   	
		
		CLP (hofteskade)                                                                                 & Legg–Calvé–Perthes disease (LCPD)                          & M91.12 \\
		Hoftefeil begge som er medfødt                                                                   & Other articular cartilage disorders, hip                   & M24.15 \\
		Plaget med rygg og knær etter fall i skibakke.                                                   & Unspecified injury of unspecified lower leg                & S89.90 \\
        skade i venstre kne \& Asperger's syndrome                                                        & Other specified injuries of left lower leg                 & S89.82 \\
		Patella femoral smertesyndrom                                                                    & Disorder of patella                                        & M22 \\
        "Slatters" på høyre kne, noe med minisken på venstre kne                                         & Juvenile osteochondrosis of patella                        & M92.40 \\
		Slatters syndrom                                                                                 & Juvenile osteochondrosis of patella                        & M92.40 \\      
		Schlatters sykdom begge knær                                                                     & Juvenile osteochondrosis of patella                        & M92.40 \\      
		beinhinnebetennelse ve,leggbein ca 1 års tid                                                     & Osteomyelitis, unspecified                                 & M86.9   \\
		kronisk beinhinnebetennelse i begge legger, skal opereres snart                                  & Osteomyelitis, unspecified                                 & M86.9   \\		
		Belastningssakde i begge knær, operert i hø. Kne den 8.10.2010.  venter på ve. Kne operasjon $^4$ & Strain of unspecified muscle and tendon at lower leg level & S86.91 \\		
		legghinnebetennelse i begge bein                                                                 & Localized swelling, mass and lump, left lower limb         & R22.42 \\
		legghinnebetennelse i begge bein                                                                 & Localized swelling, mass and lump, right lower limb        & R22.41 \\
		kronisk betennelse i venstre kne                                                                 & Localized swelling, mass and lump, left lower limb         & R22.42 \\
		Rygg og kneproblemer pga idrettsskade                                                            & Dorsalgia, unspecified                                     & M54.9 \\
		Rygg og kneproblemer pga idrettsskade                                                            & Other specified injuries of left lower leg                 & S89.82 \\   
		Rygg og kneproblemer pga idrettsskade                                                            & Other specified injuries of right lower leg                & S89.81 \\            
        

        \rowcolor[HTML]{88CC88}        
		\multicolumn{3}{|l|}{Arms and abdomen}   \\
		\hline 

		Senebetennelse begge håndledd                                      						    & Other specified disorders of tendon, left wrist  & M67.834 \\		
		Senebetennelse begge håndledd                                     						    & Other specified disorders of tendon, right wrist & M67.833 \\		
		Brokk, som gjør at lukkemuskelen til magen ikke fungerer adekvat     					    & Abdominal hernia                                 & K45 \\
		Flere år med periodevise magesmerter uten at legene har funnet årsak 					    & Unspecified abdominal pain                       & R10.9 \\
		Problemene kommer ca midt i måltidet. Smerter på venstre halvdel av magen, opp mot navelen. & Unspecified abdominal pain                       & R10.9 \\

		\hline

        \rowcolor[HTML]{FFFFFF}        
		\multicolumn{3}{|l|}{$^1$ Rheumatism is included in arthritis, no extra disease added.}   \\
        \rowcolor[HTML]{FFFFFF}        
		\multicolumn{3}{|l|}{$^2$ Possible diagnostics are not included.}   \\
        \rowcolor[HTML]{FFFFFF}        
		\multicolumn{3}{|l|}{$^3$ Unspecified in the sense that no C,D or L vertebrae is specified, although the reason for the pain is actually specified.}   \\		
        \rowcolor[HTML]{FFFFFF}        
		\multicolumn{3}{|l|}{$^4$ Only one strain remains, so only one disease is registered.}   \\		

		\hline

	\end{tabular}
    }
    

\end{table}



\begin{table}[H]

    \caption{Diseases "Other" (3/3).}	

	\tiny

	\centering

    \label{table:Table_Other_Diseases_3}
    
	\renewcommand{\arraystretch}{1.5}

    \scalebox{0.85}{

	\begin{tabular}{|lll|}
		\hline
		\rowcolor[HTML]{FFAAAA} 
		Questionary                                             & Medical                        & ICD10  \\ \hline

        \rowcolor[HTML]{FFD1AA}        
		\multicolumn{3}{|l|}{Skin diseases}   \\
		\hline   		
		
		pustolosids palmoplantaris ev dermatitt. Beinhinnebetennelse i begge legger i tre år & Atopic dermatitis, unspecified & L20.9 \\
		pustolosids palmoplantaris ev dermatitt. Beinhinnebetennelse i begge legger i tre år & Osteomyelitis 				  & M86.9 \\
		akne 																				 & Acne 						  & L70 \\
		Elveblest 																			 & Urticaria, unspecified 		  & L50.9 \\
		Kuldeallergi. Får utslett. Også når vått og fuktig .								 & Urticaria due to cold and heat & L50.2 \\
		vittiligo, hudsykdom 																 & Vitiligio 					  & L80 \\
		Atopisk eksem                         												 & Atopic dermatitis, unspecified & L20.9 \\
		hudsykdom 																			 & Disorder of the skin and subcutaneous tissue, unspecified & L98.9 \\
		irritasjon, kløe på øyne, diagnsostisert av lege                          			 & Pruritus, unspecified		  & L29.9 \\

		\hline

        \rowcolor[HTML]{FFD1AA}        
		\multicolumn{3}{|l|}{Digestive track}   \\
		\hline   		

		Morbus krons                    												   & Crohns disease						  & K50.90 \\
		morbus crohn 																	   & Crohns disease 					  & K50.90 \\
		Mb. Chron. Husker ikke når han fikk diagnosen. Var baby. 						   & Crohns disease						  & K50.90 \\
		Magesår 2 ganger. Siste påvist i forrige uke. Starter beh   					   & Gastric ulcer						  & K25 \\
		Født med øsofagusatresi (?). Ble behandlet i Tr.heim med blokking for 5 år siden." & Atresia of esophagus without fistula & Q39.0 \\
		Glutenintoleranse.  Mangler enzymer. Oppdaget ved 15 års alderen 				   & Celiac disease 					  & K90.0 \\
		Kronisk pankreatitt                												   & Other chronic pancreatitis 		  & K86.1 \\

		\hline

        \rowcolor[HTML]{FFD1AA}        
		\multicolumn{3}{|l|}{Kidneys and genitals}   \\
		\hline   		

		Kronisk urinveisinfeksjon. 					      & Urinary tract infection, site not specified & N39.0 \\
		Dårlig funksjon av lukkemuskel ned til magen      & Other disorders of urinary system           & N39 \\
		PCOS                                              & Polycystic ovarian syndrome                 & E28.2 \\
		Kronisk smertesyndrom, type 2. (nyreproblematikk) & Chronic kidney disease, stage 2 (mild)      & N18.2 \\
                
		\hline

        \rowcolor[HTML]{FFD1AA}        
		\multicolumn{3}{|l|}{General infectious diseases}   \\
		\hline   		
		
		kyssesyke ( påvist i sept 2010, syk i mai & Infectious mononucleosis, unspecified without complication & B27.90 \\
		
		\hline

        \rowcolor[HTML]{FFD1AA}        
		\multicolumn{3}{|l|}{Autoimmune diseases}   \\
		\hline   				
		
		Ideopotisk trombocytopenisk pulpura & Immune thrombocytopenic purpura & D69.3 \\
		Lupus                               & Lupus erythematosus             & L93   \\
		Autoimmun hepatitt                  & Autoimmune hepatitis            & K75.4 \\

		\hline

        \rowcolor[HTML]{FFD1AA}        
		\multicolumn{3}{|l|}{Others}   \\
		\hline   				
		
		tinitus                 											               & Tinnitus & H93.1 \\
		lettere hørselshemmet														       & Unspecified sensorineural hearing loss & H90.5 \\
		(name redacted) har et syndrom exfragilsyndrom ( usikker hvordan det skrives) $^1$ & Fragile X chromosome & Q99.2 \\
		har i ett år hatt en tumor i høyre lår som skal opereres          				   & Neoplasm of unspecified behavior of bone, soft tissue, and skin & D49.2 \\		
		Håravfall ( husker ikke diagnose)  												   & Nonscarring hair loss, unspecified & L65.9 \\		
		regnbuehinnebetendelse  for 7 år siden. og leddgikt                                & Iridocyclitis & H20 \\		
		regnbuehinnebetendelse  for 7 år siden. og leddgikt                                & Juvenile arthritis & M08 \\		
		cyste hø side i hodet, uspesifisert diagnose iflg pas 							   & Sebaceous cyst & L72.3 \\		
		Kreft i skjoldbruskkjertel. Op 9.mars 2010 										   & Malignant neoplasm of thyroid gland & C73 \\		

		\hline

        \rowcolor[HTML]{FFFFFF}        
		\multicolumn{3}{|l|}{$^1$ The original datapoint contain the name of this person. Redacted due privacy.}   \\

	\end{tabular}

    }

    

\end{table}



\begin{table}[H]

    \caption{Deleted diseases.}	

	\tiny

	\centering

    \label{table:Table_diseases_deleted}
    
	\renewcommand{\arraystretch}{1.5}

    \scalebox{0.85}{

	\begin{tabular}{|ll|}
		\hline
		\rowcolor[HTML]{FFAAAA} 
		Questionary                                             & Reason  \\ \hline

		Mulig anstrengelsesutløst astma. Har medisin til utprøving nå. Ikke fått diagnose ennå 				& No diagnose \\		
		Utredes for migrene																					& No diagnose \\		
		oppgir ikke kronisk sykdom men sier han ukentlig tar smertestillende mot hodepine, ibux eller pinex	& Taking meds but no disease given. \\		
		Oppgir at han ofte har hodepine																		& Frequent headaches, but no further description and no chronic status.\\		
		Har alltid hatt problem med knærne,spes.hø. Ble verre rundt 12 års alderen. Løse leddbånd?		    & Knee problems, but unsure why. (Maybe Schlatter since many other also have it) \\		
		Har hatt lavt stoffskifte etter virussykdom. Brukt medisin mnen bruker ikke dette i dag.			& Non chronic. \\		
		epilepsi som barn, ikke anfall på 5 år																& Epilepsy is not active anymore. \\				
		
		Har ikke astma lenger siden 6 år gammel. 															& Asthma is not active anymore.\\		
		Har hatt astma men har det ikke lengre																& Asthma is not active anymore.\\		
		
		kyssesyke i over 2 år, gått over nå																	& Two years since last episode. \\												
		Hjerteoperert som nyfødt. Går til kontroller en gang per år											& Surgery done, but no further problems.\\
		søvnproblem i en vinterperiode																		& Seasonal disease not active during questionaries. \\											
		\hline							


	\end{tabular}

    }

    

\end{table}