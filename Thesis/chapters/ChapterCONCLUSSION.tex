%*****************************************
\chapter{Conclusions}\label{ch:conclussions}
%*****************************************


% Restate the main research questions and objectives of the thesis.

The main objective of this thesis was to explore how social influence different aspect of health. We did that showing that is relevant for several topics, such as bacterial infection (Paper A), obesity (Result I and Result III), inflammation (Result II), vitamin D levels (Paper B), and medication usage (Result IV).

% Summarize the main findings and contributions of each paper, highlighting their novelty, significance, and relevance to the field.

In Paper A we show an increased risk of transmission between 3.5\% to 5\%, and in particular, women are more at risk of social transmission. We also showed that SPA-type specific strains spread regardless of social context, while carriage is more spread in the school network, and additionally also physical and overall with the enrich definition. This updated the population's prevalence of \gls{staph}, showed new methods to calculate social influence, and put emphasis on how to better prevent \gls{staph} transmission at schools which can also extrapolate to other contexts.

Result I and Result III show how students with higher BMI tend to cluster together, and we show that the total number of friends based on BMI levels is the same importance or more as other variables such as smoking or PA; also showing that increased connectivity with overweight and obese increase BMI in the future, while increased connectivity with healthy decreases BMI. Other works have proposed different techniques to decrease BMI; here we found that increasing connectivity is not good enough, and it should be increased with a particular group and not the whole group. We also applied machine learning with both host factor and social data, as a new way to compare both to discern which has a bigger impact on BMI.

In Result II we showed that clusters of friends tend to have similar inflammation markers, and this similarity increases as the academic years progress further emphasizing the social influence regarding these variables. Previous work showed that social isolation influences inflammation processes; our main contribution is showing how the opposite is also true.

In Paper B we showed that friends tend to share similar levels of vitamin D even during the winter night and that women in particular tend to influence other women into adopting similar solarium habits. This opens new health guidelines to increase vitamin D in the general population.

 %   Discuss the strengths and limitations of the research, acknowledging any methodological or conceptual challenges, but emphasizing the overall rigor and coherence of the thesis.

Some of the limitations in this work include the difficulty of gathering social data ethically, and that we can't tell which type of influence is due people to people or just by sharing the same environment. In Result II we can't tell if friends influence each other into having health habits that improve or worsen their immune reaction, or if all of that is irrelevant and they are just sharing a very dirty or clean school. Also in Result II, we showed influence related to inflammation exists, but the meaning of having the particular shared marker is difficult to interpret. In Paper B we can't tell how much people share the same diet or PA, or how much they can influence each other on this topic, and to do so we need to collect more rigorous data. Also in Paper B, we didn't check friendship against bone mineralization which is very related to vitamin D levels. Some methods used in Paper A need to review the mathematics more rigorously, and in particular improve the R libraries giving non-inverse matrices errors when the system solution should, in theory, be solvable. In Result III the ML hyper-parameters were not optimized and better models, with better explainability, should be easy to find. Result IV extrapolates the reasons for self-medication from previous Norwegian studies, but we don't ask the students directly. We also discussed in the mathematical background how the visualization of the network can be very challenging, and is not straightforward to create a figure able to convey the pattern in the relationships. 

%    Reflect on the broader implications of the research, both in terms of theoretical insights and practical applications and suggest possible avenues for future research.

%% In the Discussion chapter, we discussed plenty of examples in which research can be expanded, and in some cases, we have already generated interesting results.

The advance of antibiotic resistance bacteria seems to be a pressing matter at the moment which will require a better understanding of how people share the same environment or transmit the bacteria to each other. A better understanding of how a cluster of friends influences diet and PA is also a very interesting branch of research in this context. Social studies can also collaborate with public health to investigate how to increase connectivity in obese people so they adopt better dietary habits. Finally, precision in social health interventions targeting specific groups better rather than broad reaching can be done and increase the effectiveness of such interventions while reducing the cost.
  
% Conclude with a concise and memorable statement that captures the essence and value of the research, and its potential impact on the field and beyond.

In summary, this thesis provides evidence that suggests social influence is currently present in many topics, and accounting for this factor is as important, if not more, than accounting for other classical variables such as smoking or PA. We have also sadly experienced recently the necessity to account for social interactions during the SARS-Cov-2 pandemic, which also demonstrated a good example of why this type of analysis can be critical.




%% Finally, we also discussed how this is extendable to both health and non-health topics; showing also that unfortunately, this type of analysis can work against the general public interest, and showing the necessity of being prepared to defend ourselves from external influences, and is no better way to do so than with transparency and public research.