


\section{Introduction}


\section{Vitamin D metabolism }

Vitamin D, along with vitamins A, E and K, belong to the fat-soluble group, they are absorbed through the intestinal tract inside lipids droplets. Vitamin D can be stored inside liver’s fat tissue for about 2 months and be released on demand. There are two methods by which the body obtains Vitamin D, via sun exposure (pre-vitamin D3) and food intake or dietary supplements (vitamin D2 and D3) in the small intestine via passive diffusion using intestinal membrane carrier proteins [D8]. 

Regardless of intake method, vitamin D needs to undergo two hydroxylation processes. The first one occurs in the liver and transforms vitamin D into 25-hydroxyvitamin D, 25(OH)D, using 25-hydroxylase. The second hydroxylation happens in the kidneys and forms 1,25-dihydroxyvitamin D ,1,25(OH)2D, using 1-α-hydroxylase, which is the final and physiologically active form of vitamin D. This form is known as calcitriol. If calcitriol becomes excessive, then it is converted to 24,25-dihydroxycholecalciferiol, which is less active.  

Both vitamins D2 (ergocalciferol) and D3 (cholecalciferol) raise 25(OH)D levels, the metabolism and actions of vitamins D2 and D3 are identical, and they only differ in the chemical morphology present in their sidechain structure. Evidence suggests that vitamin D3 increases 25(OH)D levels greater and longer than vitamin D2 [D25]. The other 3 forms vitamin D, D1, D4, and D5 are not relevant in this study. 

There are two primary functions for vitamin D, to absorb calcium and phosphate from the guts into the blood; and to inhibit parathyroid hormone (PTH) production. When calcitriol is present in the blood, it stimulates the epithelial cells in the gut to increase production of calbindin-D proteins. This increases the absorption of calcium from the brush border to the basolateral membrane, where calcium finally enters the bloodstream. 


\section{PTH and Calcitonin}

PTH maintains healthy levels of calcium in the blood serum. If calcium levels are too low, then PTH binds to osteoblast, which increases RANKL, which transforms pre-osteoclasts into osteoclasts. Osteoclast breaks down bones apart in order to maintain calcium ions levels as they should be. PTH also increases renal re-absorption in the distal convoluted tubule (DCT). Ultimately, this calcium is reabsorbed into the blood, but in exchange, you will urinate more phosphate. PTH also increases the production of vitamin D in the kidney, by increasing the production of 1-α-hydroxylase. PTH also inhibits the production of PTH, making a negative feedback loop. Malignant tumors can also secrete PTH peptides (PTHrP) that mimic PTH which help tumors grow by increasing calcium levels. PTH can also lead to the increase of IL1, IL2, IL6 and INF-ɣ , which increases osteoclastogenesis that leads to further bone density loss [(Zanchetta et al., 2016; Micic et al., 2019)]. 

Calcitonin is secreted in the thyroid glands. Calcitonin simply decreases calcium levels by inhibiting osteoclasts. Having high calcium levels in the blood serum is not as dangerous compared to having too little, thus calcitonin has less homeostasis power than PTH and calcitriol. 


\section{25OHD standardization}

Antibodies and chromatography are the two main methods to assess 25OHD levels. Each has several sub-methods that give high variability in the final 25OHD levels [D12, D13, D14]. As a result, comparing studies is challenging. The aim of VDSP is to create a global partnership that aims to standardize the laboratory assessment of vitamin D levels. By ensuring that the serum total 25OHD measurements are uniform and precise, the program seeks to enhance the identification, assessment, and management of vitamin D insufficiency and deficiency across different regions, times, and laboratory methodologies [VDSP] 



\section{Vitamin D sources}


UVB radiation 

 

UVB radiation can come from the Sun or tanning beds. It has several detrimental health effects, including "thymine dimer" DNA damage with the subsequent sunburn, hydroxyl and oxygen radicals, melanomas due melanocytes DNA damage, vitamin A depletion due to keratinocytes and melanocytes DNA damage. UVB always comes along with UVA radiation in about 5 to 95 proportion; UVA offers no health advantages, and it has more penetration power. UVA causes collagen breakdown, free radicals, vitamin A depletion, cataracts, and suppression of T lymphocytes and apoptosis which results in the proliferation of melanoma, basal cell carcinoma, and squamous cell carcinoma.  Sunscreen can protect against UVA and UVB, however the sun protection factor (SPF) only describes UVB blocking, and the lotion can be deceivingly marketed as "Sun protection" when in reality it offers no UVA protection at all. There is no standard for UVA protection labeling and even the SPF standard changes from country to country. Darker skin lowers UVB absorption [D9]. Skin color is mostly determined by melanine, and slightly by carotenes and iron levels in blood. Melanine is determined by genetics and increases slightly by the amount of UVB exposure. 

 

Food 

 

Foods with natural source of vitamin D are almost exclusively concentrated in the fat fish group. Namely mackerel (1010 IU D2+D3/100gr, 168% RDA [D66]), raw salmon (435 IU, 73%) or caviar (117 IU , 20%). White fish has very little amount in comparison, such as cod (36 IU, 6%), however fish composite where the whole fish is mixed including liver or kidneys, has increased amounts such as canned tuna (269 IU, 45%). Meat and poultry also have a small amount, beef (28 IU, 5%), bacon (9 IU, 2%) or chicken (9 IU, 2%). Dairy is a mixed bag; egg yolk is rich in fatty acids and contain high levels, while the white part does not, in average a whole raw egg contain about 91 IU D2+D3/100gr, 15% RDA; while whole 3.5% milk has near zero (2 IU, 0%). Vegetables, fungi, fruits and nuts also contain no natural vitamin D. However, mushrooms grown under artificially high UV light contain about 1140 IU D2+D3/100gr, 190% RDA. [D1] Food preparation decreases by about 10% the amount of vitamin D contained in their raw format counterpart. 

 

[[Furthermore, in the last 70 years food trends have changed from healthy fresh meals for quick processed frozen food with high caloric input and low nutritional value.]] 

 

Supplements 

 

Vitamin D supplements are widely available in supermarkets and pharmacies as over the counter drugs; these versions already contain more than the daily recommendation doses of vitamin D. Higher concentration pills requires medical prescription. Hypervitaminosis D is almost exclusively occurring due high intake of dietary supplements [D45-D47] 

 

Diets that avoid animal-based products require supplementation. However, some supplements are also animal based. Vitamin D2 is obtained by irradiating ergosterol in yeast, while vitamin D3 can be obtained with mushroom powder, irradiating pre-vitamin D3 from lanolin which comes from the wool of sheep, using a similar process with lichen instead, or directly extracting oil from fish liver. 

 

Epidemiology 

 

There is an ongoing shift in demographics in first world countries that are predominantly of Caucasian ethnicity with people of middle easter, black, and south Asian sea descent who require higher UVB to synthesize vitamin D [1] [3] [Meta-Migrant low D].  The prevalence of vitamin D deficiency (25OHD < 50nmol/l) is significantly elevated around the world [D33-D36]. Tromsø (19% prevalence) have a low prevalence compared to Norway (28%), which also have a lower prevalence compared to Europe (40%) and other European countries: Greece (62%), Germany (44%), Netherlands ( general population 29%, 68% in male and 53% in female Chinese immigrants [Chinese Netherland]), Ireland (27%), UK (57%, 94% among Bangladeshi background [BangladeshMigrant]), Denmark (general population 37%, foreigners in clinical setting 80% [DanishMigrant]), Finland (64% foreigners, 7% nationals), Iceland (34%), Estonia (29%) [Estonia].   Italy ( 77.5% foreign children [ItalyMigrantChildren] ) Switzerland ( 86% foreigners [SwissMigrant]) . In other parts of the world, we have Canada (37%), USA (24%), Australia (23%), in the Asia region (58%) or Africa region (34%). 

 

Vitamin D has been related to several diseases, such as bone related diseases, cancer, cardiovascular diseases, depression, multiple sclerosis, diabetes and obesity. It's also linked to modulation of inflammatory processes and testosterone production. However, except for bone related diseases where there is somewhat strong consensus, there is contradictory evidence between vitamin D levels and the rest of health issues. This might be due to 25OHD tests not being properly standardized and poor experiment design [D13, D34, D67]. 

 

Bone Health 

 

Vitamin D deficiency led to a lower amount of calcium and phosphate in blood, resulting in abnormal bone grow, rickets, brittle bone diseases, osteomalcia, or osteoporosis [D68]. 

 

Rickets is a rare disease in developed countries that increase significantly from 10% to 70% in places of Africa, Middle East, and Asia [D69] , and has higher prevalence in recent year than in the pass, up until the introduction of vitamin D and Calcium supplements in infant food [D69-71]. Most forms of rickets are due to vitamin D deficiency in children, while severe rickets cause developmental delay, hypocalcemic seizures, tetanic spasms, cardiomyopathy, and dental abnormalities [D70, D71]. Almost all patients with rickets had been partially or exclusively breastfed [D70, D72]. There are different definitions and diagnostic criteria to define osteomalcia, but prevalence is about 3.7% with higher rate in women [D73]. 

 

Osteoporosis is characterized by the deterioration of bone tissue leading to bone fractures. About 15% of people older than 50 and 70% older than 80 suffer osteoporosis worldwide [D74]. Osteoporosis is caused by the lack of calcium intake, while osteomalcia and rickets are caused by the lack of vitamin D intake; however, a lack of vitamin D intake contributes to the severity of osteoporosis [D9]. For the general older adult population, some studies show that vitamin D and calcium supplementation increase slightly bone density, display reduce fracture rates. [D9, D75-D77], while other studies show no difference [D77-D84] In the United States, bone density, mass, and fracture risk are correlated with 25OHD in white individuals and Mexicans but not for black individuals [D18, D85, D86] 

 

Muscle Weakness 

 

Vitamin D assists in the development of muscle fibers. A proper support of the bone structure is needed for optimal bone health, so indirectly, vitamin D also helps bone development not only via calcium absorption but also due muscle growth. Inadequate levels of vitamin D can lead to myopathy. Sadly, experiments with vitamin D supplements also include calcium supplements. As such, being able to differentiate between the advantages of calcium and the advantages of vitamin D independently is challenging. Studies are also not standardized in the amount of nutrient or the time in which they are administered; this is important as nutrient absorption varies with time of the day and with the food that you are eating. For example, calcium inhibits iron absorption, while vitamin C enhances iron absorption [D87, D88]. Another example is that proteins need carbohydrates to be absorbed, vitamins can be water or fat soluble, and each group require different food setup to maximize absorption; all of these variables are rarely taken into account during the intake of supplements in studies. 

 

For a general population, studies have shown either inconsistencies [D89] on the effects of vitamin D supplementation with respect the muscle strength and muscle decline, or no correlation [D90, D91]. 

 

Cancer 

 

Some studies suggest that vitamin D inhibits carcinogenesis and slows tumor growth. They also suggest that it has anti-inflammatory effect, can modulate the immune system, it triggers cells to proapoptotic, and is antiangiogenic [D9, D92]. 

 

Observational studies show correlation between low 25OHD and cancer mortality [D77, D93, D94]. Some clinical trials support the hypotheses of these observational studies [D95-97]. And a particular clinical trial supports that vitamin D supplementation in a general population delays cancer appearance by 5.3 years in median [D98]. Other suggest no correlation effect [D77, D99]. In both cases, there is a confusing wording in the effect of vitamin D or calcium supplementation, suggesting that individuals with low 25OHD have an increased risk of cancer mortality that is fixed by supplementation, while individuals with normal 25OHD levels have no increased risk and thus the supplementation is useless because no association is linked between supplementation and cancer mortality. This makes confusing the effect of supplementation and mortality because it is not the supplementation itself what reduce the cancer mortality but having appropriate 25OHD levels. However, we have already established that low levels are cured using supplementation provided that you don't suffer for any of the co-morbidities that prevent 25OHD absorption. 

 

In breast cancer there is contradictory evidence, from inverse levels 25OHD and mortality, to the opposite, passing through no correlation. [D100-D106]. Lung cancer shows no association between circulating concentrations of vitamin D and risk of lung cancer [D107]. Pancreatic cancer: No correlation and positive levels of 25OHD associated with higher cancer risk [D108-D110]. Colorectal cancer: Inverse levels 25OHD and mortality specially in women, no correlation, and positive correlation between vitamin D supplements and calcium supplements with the development of polyps [D100, D111-D113]. Prostate cancer: Contradictory evidence between 25OHD levels and cancer risk, mortality, and length [D114-D122]. 

 

Cardiovascular diseases 

 

In the context of cardiovascular diseases (CVD), vitamin D regulates the RAS [D123] (renin-angiotensin-system), which regulates blood pressure, systematic vascular resistance, electrolyte balance and fluid. Also regulates vascular cell growth, fibrotic pathway, and inflammatory pathways. Indirectly, vitamin D enhance the absorption of calcium and potassium, which are both elemental pieces in the cardiac action potential for both pacemaker cells and contractile cells, specially the later as the absolute refractory in contractile cells is essential to prevent tetanus. Vitamin D deficiency has been shown to be associated with vascular dysfunction, arterial stiffening, left ventricular hypertrophy and hyperlipidemia [D124]. 

 

As such, observational studies have linked vitamin D with lower CVD risk. A positive association between high 25OHD and lower CVD events (myocardial infarction, ischemic heart disease, heart failure, and stroke) and mortality [D125]. There is also a study that shows an association between high and low 25OHD levels and CVD [D126]. And other studies found correlation between low 25OHD and high CVD [D127, D128].  However, clinical trials contradict all of this [D77, D98, D129], with some studies suggesting that it protects against cardiac failure, but not against myocardial infarction or stroke [D130]. 

 

Supplements have also shown to reduce total cholesterol and LDL, but not HDL [D131], they have contradictory effect on blood pressure for normal weight patients [D132, D133], and when taken with calcium increase blood pressure in overweight and obese patients, while having low 25OHD also increase blood pressure [D132, D134]. 

 

Depression and other mental illnesses 

 

Vitamin D receptors are present in several areas of the brain, and it is believed to be involved with depression, dementia, schizophrenic-like disorders, hypoxic brain injury, and other mental illnesses; as well as neuronal development and a decrease of microglial inflammatory function. [D135-D139]. 

 

But once again, clinical trials found that the administration of vitamin D supplements were not linked with reduction of depressive symptoms [D140-D144]. It is worth mentioning that none of these studies tested the combination of vitamin D supplements, plus antidepressant, in individuals with low 25OHD. 

 

Multiple sclerosis 

 

Multiple sclerosis (MS) chronic disease of autoimmunity or oligodendrogliopathy origin [D145] where an inflammation of the cover of nerve cells in the brain and spinal cord disrupt the ability of the nervous system to transmit signals properly. As a result, it can present almost any neurological symptoms. MS occurs less frequently near the equator and more frequently near the poles [D146]. This led researchers to investigate whether sun exposure has anything to do with MS, which is of course related to vitamin D absorption [D77]. 

 

Studies have shown the presence of low 25OHD before and after MS begins [D146-D149]. Others show that normal 25OHD levels reduce risk of contracting MS and decrease the time in between relapses once it has started [D150]. But clinical trials once more contradict these findings [D146, D151]. 

 

Diabetes and glucose homeostasis 

 

 Vitamin D stimulates the secretion of insulin via vitamin D receptors on the pancreatic beta cells, which reduces insulin resistance, and it is also involved in the insulin signaling pathways [D152-D154]. Some studies have shown a relationship between vitamin D and risk of diabetes [D9], others have shown an inverse correlation between vitamin D and blood sugar [D155]. 

 

Clinical trials tell that there is no relationship. Including vitamin D not improving insulin sensitivity in overweight and obese patients [D156], supplements not having an effect in glucose homeostasis or insulin resistance [D77, D157], no prevention of going from pre-diabetes to diabetes [D158] , the same for overweight or obese with normal 25OHD levels, but yes for those with lower 25OHD levels [D152, D159], and no benefits on individuals that already have diabetes [D154]. 

 

Obesity 

 

As vitamin D is fat soluble, more fat deposits in the body will sequester more vitamin D. Observational studies suggest that weight has an inverse correlation with 25OHD levels [D41, D42, D160, D161], and even reduce weight gain in postmenopausal women [D162] but vitamin D supplementation doesn't help with weight loss [D163]. 



 
\begin{comment}


...instead of pushing for open source / free software. Free software is often less expensive or completely free than proprietary counterparts. Selling your soul to MS. Free software provides transparency and greater control over what the software is doing on your system; this issue alone should be enough to ban the use of any medical data with anything that is not open source product.O pen standards, making it easier to share data between different applications. Free software is based on the ethical principle that software should be freely available to everyone

Learning curve: Free software may have a steeper learning curve for users due to customization options and complexity.



There is no valid in place solution to share results between collaborators. Talk GIT.


Also, OneDrive doesn't support anonymous link sharing. For any external collaborator, or simply curious minded person who want to see the results of the analysis, you need to ask permission that need to be granted manually (by me), and in order to do so you need at least a mail account. But you need a Microsoft mail account if you want to access the full range of functionality. Lastly, due security issues, you also need a UiT account since we have a system wide ban on external collaborators sharing documents without an organization mail account.

OneDrives can comply with UiT's privacy of data policies, however, further configuration is required via Azure in order to setup a proper secure server for your project. Dropping red and black data in a common OneDrive folder is not allowed.


\subsubsection{TSD}

\href{https://www.uio.no/english/services/it/research/storage/filesender.html}{Services for sensitive data (TSD)} is a platform for collecting, storing, analyzing and sharing sensitive data in compliance with the Norwegian privacy regulation. TSD is used by researchers working at UiO and in other public research institutions (the UH-sector, universities, hospitals etc.). The TSD is primarily an IT-platform for research even if in some cases it is used for clinical research and commercial research.

TSD has the advantage that can be use a virtual remote machine, which is quite convenient if you don't want to be bothered with security issues related to data privacy. This include the possibility of several people working on the same documents (without version control software) and a common space to share results securely.

The main disadvantage of TSD is that getting something out of TSD is also a bureaucratic ordeal. For example, if you want to write an article, and you want to include a figure you generated in there, you might have to wait several weeks before you are able to extract the figure from the server; basically until somebody look at it manually and check that is not violating anyone privacy. And this will repeat each time you generate any new file. Even extracting your own code is time consuming task because you also need to check that the code don't contain privacy leaks.

Other minor inconveniences is that, at the time of writing this, R software was limited to Windows architecture only. You also need to pass periodically another bureaucratic layer to keep access to your project, as project expire over time.

TSD however has the potential to become the best cloud-sharing option, by far. It already has in place the physical infrastructure, with the hard drives with the actual data within Norwegian borders, and software that allow for the execution of a remote virtual machine which you can use anywhere you want. It just need to include it own private GIT system where you can store and importantly retrieve the parts of your project that are not red or black data.

The solution that the people in Sommarøy were telling about, comment here




%To convert all these original data, into a more human and computer friendly version, I use the script "dataCleaning.R". Once the data is "clean" and we filter out the values that we don't want, then we can proceed with our analysis.

%In total, among these 5 files, we have available 944 columns that are unique. Which we later on transform into about 656 variables, with that number increasing depending on whether you consider each individual network as independent datapoints or all part of the same dataset.


\subsection{Data redundancy}




\subsection{Second hand obesity}

Society plays an instrumental factor in how many opportunities an individual have to gain weight. Obesity is not a transmittable disease, and is not possible for a person to eat and make another person gain weight, and yet people can influence your obesity in several ways:

    \begin{description}
    
        \item[Peer pressure:] People often eat and drink more when they are with friends or in a social setting. Peer pressure can make them indulge in unhealthy food choices that can contribute to weight gain.

        \item[Marketing and advertising:] The food and beverage industry spends billions of dollars each year on marketing and advertising. This can influence people's eating habits and preferences, leading to the consumption of calorie-dense and nutritionally-poor food options.
        
        \item[Family and cultural norms:]  Cultural values and familial customs often shape an individual's eating behavior and dietary habits. Family members and cultural traditions may promote high-fat, high-sugar, and high-salt diets that can lead to weight gain.
        
        \item[Social norms and attitudes:] In some societies, obesity is viewed positively, and a person who is overweight may be seen as wealthy or healthy. This can lead to individuals feeling less motivated to lose, or control, their weight.
        
    \end{description}

Is the individual ultimate action of high energy intake to low energy output the responsible for obesity levels. 


\section{High influence (wording) }

\section{Limitations}

\subsection{Nutritional data sucks}

There are some conditions which makes obesity hard to manage are not influence by friends.  

Hypothyroidism lower metabolic rate decreasing energy output. It also impair kidney function leading to unbalance fluid filtration, but this edema is due water and salt, not fat.

Individuals with Prader-Willi syndrome have an uncontrolled and insatiable appetite, leading to excessive weight gain.

Conditions in which the adrenal gland is affected lead to an unbalance of hormones, in particular cortisol. Cortisol stimulate breakdown of adipose tissue, but also suppress ability to maintain muscle mass which leads to slower metabolism. This is the case of Cushing's syndrome or Congenital Adrenal Hyperplasia.

Leptin stimulate the brain regulating the feeling of hunger. The more leptin you have, the more full you feel and the less you eat. Leptin levels increase as fat mass increases and viceversa. But if the brain is constantly overstimulated by leptin you get leptin resistance, causing more hunger feeling leading to overeating. As obesity increases leptin, obesity lead to leptin resistance, leading to more obesity in a positive feedback loop. Conditions that affect leptin are Congenital leptin deficiency. Leptin increses with emotional stress, sleep apnea, estrogen, and obesity. Decreases with PA, and testosterone.

Any cause that leads to insulin resistance, such as Polycystic ovary syndrome (PCOS), Insulinoma, or type II diabetes.








\section{The Need for Up-to-Date Skills and Tools among Health Professionals}


\subsection{Programming languages}

\subsubsection{ LaTeX }

\LaTeX \cite{ref:latexintro} is a system for preparing written documents. The main characteristic is that, unlike in Microsoft Word, LibreOffice, and other similar software where you see the final result of what you are written live as you edit it, in LaTeX the user types in plain text keeping style and content separated, in a similar way as HTML+CSS works. Later on the code is compiled into a PDF document where you can see the final stylized result.\vspace{3 mm}

The main inconvenient of LaTeX is the learning curve, however LaTeX is widely used in all academia fields for the communication and publication of scientific documents. In my opinion, the inconvenience of having to learn LaTeX outweighs, by far, the amount of time that you are going to save editing documents in a "What You See Is What You Get"  \cite{ref:wysiwyg} editor. To the point that mathematics communications alone would be near impossible to perform without this software.\vspace{3 mm}

As opposite to Microsoft Word where you can synchronize automatically with OneDrive, Latex doesn't have by itself a collaborative interface and you are dependent of using a control software, such as GIT {\tiny [\ref{sec:git}]} , or using online services, such as Overleaf {\tiny [\ref{sec:overleaf}]}. In any case, this problem can also be adverted by setting automatic version control scripts; which to be fair, scare people outside mathematics, informatics, and physics fields.\vspace{3 mm}

\subsubsection{SPSS}

The letters SPSS \cite{ref:spss} stands for "Statistical Package for the Social Sciences". Is a proprietary statistics software developed for social sciences researchers who have very limited, or none, knowledge of programming.\vspace{3 mm}

The advantage of SPSS is that is composed of easy to use drop-down menus. If a person has some basic knowledge of statistics, SPSS is very easy to use even for complex multivariate analysis.\vspace{3 mm}

The shortcomings are that you can't do extensive scripting in SPSS, so anything that we do in this project, such as generating results automatically, generating latex code automatically, or generating websites automatically, won't be possible. During the course of the project we also tend to change biological definitions, such as, what is a teenager, or what does it mean to be a carrier of a bacteria. Running manually all the drop-down menus in SPSS, again and again, each time we decide to change the definition of a disease, or who is the target population to analyse, would be an insurmountable time consuming task. As such, is impossible to use SPSS in this project.\vspace{3 mm}

Other negative issues are the proprietary license cost, worsened by the fact that it has a software as service license, plus the issues with close software security. Adding to this, there are statistical method within several libraries available in R that aren't in SPSS, namely almost everything that has to do with network analysis. For fairness, mention that it has a trialware license where you can use the software for free, but I would strongly advice against wasting your time using it given all the limitations that presents.\vspace{3 mm}

Some of the Fit Future data is sadly only available in SPSS files. However there are plenty of libraries that overcome this so you can read the data outside SPSS without having to do lengthy transformations.\vspace{3 mm}

\subsubsection{R}

R don't have classes. The classes that R provide are to classes, the same that the Frankenstein's monster is to humans.

R \cite{ref:rproject} is a programming language specialized in statistical analysis. Is very popular among the data analysis community and praised everywhere. But not by me. R has a lot of shortcomings and limitation as a programming language, which I discuss in great detail at the end of this document.\vspace{3 mm}

In essence, R an evolution of the APL programming language (1966) \cite{ref:APLlanguage}, that later evolved into S (1976) \cite{ref:Slanguage}, which latest evolved into R (1993). Because of this R has evolved dragging the plenty of disadvantages of it predecessors, and is a nightmare for anyone who has learned to do object orienting programming (such as with C++). To the point that, even the most basic of operations which is the assignment operator, is nowadays a subject of debate as whether is appropriate to use "<-", "=", ":=", or "<<-",  \cite{ref:Rassignoperators}, depending on what environment you want your variable to be saved. This is an unnecessary complicated concept for anyone who understand the classic concept of scope of a variable, which it is a very simple concept used in any other programming language. It doesn't have a proper class structure, it doesn't have enumerations, it mixed access operators ("[]" for vectors, "[[]]" for lists), it doesn't have polymorphism, it doesn't have pointers or any type of direct memory access commands for optimization, you can't make function that distinguish between passing as argument, reference or constant reference, it doesn't have constants variables, it doesn't allow for variable type declaration, boolean logic involving NA values defy previously well established programming patters (R define FALSE and NA = NA ; which is not true, FALSE and anything has always been FALSE since George Bool was born), it doesn't have switch structure, and many more.\vspace{3 mm}

R is incredibly inefficient, to the point that you need to use the Rcpp \cite{ref:rcpp} library and throw your data into a C++ program that runs your heavy operations with a big dataset if you want your code to work in an acceptable amount of time. But none of this matter in comparison with the amount of bugs and difficulty debugging your code that R has in comparison with other languages. And that is what makes, in my opinion, the language inappropriate for scientific research.  The general population are not well trained to proper testing functions; not even IT engineers do proper testing! and as such, there must be hidden bugs everywhere nowadays that influence the outcome of studies.\vspace{3 mm}

The only advantage of R is that is easy to learn and good for tiny projects, which makes it very popular. But just to remind the reader, Java has been the most popular programming language in the last 20 years according to the TIOBE index \cite{ref:tiobeweb} \cite{ref:tiobewiki}, and is another good example that popularity doesn't correlate with quality.\vspace{3 mm}

Another advantage is that R has plenty of libraries for statistical analysis. This is not a competitive advantage anymore as Python is easier to learn, has already all libraries that you would ever need, but doesn't have as many problem and disadvantages that R has.\vspace{3 mm}

As a personal advice, if you need a general purpose programming language, learn to program with C++, and if that is too complicated for you, use Python instead. And if you want a language that is very specialized in mathematics use MATLAB. And if you need something that is even more niche, and you need one particular area of mathematics use PROLOG for example. But don't go into the R rabbit-hole (much more less SPSS of course).\vspace{3 mm}

\subsubsection{C/C++}

While C++ would be my preferred program of choice, it is not the most popular language for data analysis, in part for the initial learning curve that has over R or Python. There is some work done in C++ though, in particular generating websites.\vspace{3 mm}







%*****************************************
\section{Social influence on infection diseases}
%*****************************************


The development of a lens with enough power to observe microbes was pioneered by Antonie van Leeuwenhoek (1632-1723), while Louis Pasteur (1822-1895) and Robert Koch (1843-1910) made significant contributions to our comprehension of the hidden realm of microbes. However, humanity has known early notions of diseases since ancients civilization and that they were somehow transmitted around the air, and getting close to a sick person was a bad idea and a likely way of getting infected.

Infectious diseases can be transmitted through social interactions in 3 main ways:

    \begin{description}
    
        \item[Direct physical contact:] When people come into direct physical contact with one another, infectious diseases can be transferred through skin-to-skin contact, saliva, blood, or other bodily fluids.
        
        \item[Indirect physical contact:] Infectious diseases can also be transmitted when people come into contact with objects or surfaces that have been contaminated with the pathogen and then touch their mouth, nose, or eyes.
        
        \item[Airborne transmission:] Infectious diseases can be spread through coughing, sneezing, and talking, as respiratory droplets can contain the pathogen and infect others who inhale them.
        
    \end{description}

It is no surprise to find that \gls{staph} or any other pathogen is transmissible via social influence in our population. However our contribution was to be able to put numbers to it and being able to tell how much.



%*****************************************
\section{Friends are the sunshine of life}
%*****************************************

Similar to obesity, people tend to do similar activities of those of their friends. You might be sitting on front of the PC playing videogames all day long by yourself or with other friends and never see the sunlight, or you can become a human lobster by suddenly going in a vacation in the middle of July to Costa del Sol with your family. Both cases are unhealthy extremes examples, but nevertheless illustrates how your behaviour influence UVB exposure and vitamin D synthesis.

Furthermore, we already discussed how society can dictate your nutritional habits, which may or may not be rich on D2+D3.


Contraceptives


But once again, it falls on the patient responsibility to guard himself from the Sun, and eat properly.




\section{The data cleaning process}



\section{All programming languages lack a higher analysis abstraction}



\section{Automatic analysis}




%*****************************************
\section{Interdisciplinary field}
%*****************************************

My background is computer science. As such I started having little to no idea of immunology, microbiology, endocrinology, and so for. And my knowledge on statistics and graph theory was probably lower that it should has been. Working and learning from so many topics at the same time has been a challenging work.

Furthermore, the need for many collaborators from such fields will arise sooner or later. And having to find a time that suits everyone involved has been proven to be an almost insurmountable task of nobody's fault other than the lack for humans to be able to work for more than 24h a day. This leads to progress being achieve at a very slow pace in each of the topics individually. For example, we had the biomarkers and obesity results on stand-by for about 1 year and half, and the vitamin D results for almost a year. My recommendation for future researches in a similar position would be, try to work at everything at the same time and overlap waiting times as much as possible.



% https://www.scribbr.co.uk/thesis-dissertation/discussion/

\begin{comment}
To improve the discussion chapter, consider the following:

Identify common themes: Look at the discussions in each paper and identify common themes or findings that are relevant across multiple papers. This will help you identify the big picture implications of your research and tie the individual papers together.

Offer a synthesis: The discussion chapter should offer a synthesis of your research findings. This means that you should connect the dots between different papers and draw conclusions about what your research means as a whole.

Provide a comprehensive analysis: Consider expanding beyond the direct implications of your research and providing a comprehensive analysis of the field or discipline in which your research is situated. This can provide context for your research findings and help readers understand the broader implications of your work.

Address limitations and future directions: In addition to discussing your findings, it is important to address the limitations of your research and provide suggestions for future directions for research in your field.

Consider the audience: Keep in mind the audience for your discussion chapter. It should be accessible to a broad range of readers and written in clear, jargon-free language.

Overall, the discussion chapter should tie the individual papers together and provide a comprehensive analysis of your research findings. It should offer a synthesis of key themes and draw conclusions about the significance of your research within the broader context of your field or discipline.
\end{comment}