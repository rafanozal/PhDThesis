%*****************************************
\chapter{Aims of the thesis}\label{ch:aims}
%*****************************************

%\colorbox{PaperColor}{\textcolor{black}{Paper}}
%\colorbox{ResultColor}{\textcolor{black}{Result}}


% S - Who and What

The overarching objective of this research is to develop methodologies and conduct exploratory studies to  evaluate the impact of social influence on a range of health-related topics. This work targets several topics and is done with the help of an interdisciplinary team of health professional researchers. In parallel, this work aims to provide a framework that enables researchers to facilitate faster analysis and produce improved visualizations. 

% M - By how much

The results of these studies must be a quantifiable measure of how much measuring or modifying the social networks could benefit the studied population. Secondarily, speculate how these results could affect the general population and the advantages and disadvantages of influencing and changing their social network.

% A - How?

The first step of our research is to study how infections and the immune system behave in the population. First, by measuring the spread of \gls{staph} \colorbox{PaperColor}{\textcolor{black}{(Paper A)}} and investigating the possibility that inflammatory processes may be similar across individuals or schools \colorbox{ResultColor}{\textcolor{black}{(Result II)}}. The second step is to inspect the social aspect of obesity with classical methods \colorbox{ResultColor}{\textcolor{black}{(Result I)}} and using machine learning models \colorbox{ResultColor}{\textcolor{black}{(Result III)}}. Lastly, we look into other variables of interest such as how friends influence vitamin D levels \colorbox{PaperColor}{\textcolor{black}{(Paper B)}} or medication usage \colorbox{ResultColor}{\textcolor{black}{(Result IV)}}.

For our secondary objectives, we want to present tutorials and proof of concept on how to apply \gls{sna} techniques. For this, we choose prescriptions and drug interactions \colorbox{PaperColor}{\textcolor{black}{(Paper C)}}. We also want to present a user-friendly wrapper library to abstract away the complexity and details of \gls{sna} and other statistic and machine learning models, where biases are checked automatically, fundamental analysis reports are generated with minimal intervention from the programmer, and plots or figures follow basic design rules for good visualization.

% R - Why?

Starting with infectious diseases is a good starting point as it is the classical topic for understanding the network structure, and testing and developing new methods. It also has the advantage of updating targeted interventions in similar populations in the future. Later on, obesity in particular is a good topic for \gls{sna} due to having a substantial impact on individual lifestyle choices, including diet, exercise, and weight management; all of which are influenced by friends. Lastly, topics such as vitamin D levels are purely explorative and we want to determine if there was a connection with the social network.

Regarding our programming objective, wrapper libraries serve as a layer of abstraction that simplifies the usage of lower-level functionality, allowing programmers to build applications more efficiently and with less effort. We also want to extend these advantages and make a high-level abstraction in the statistical context.

%The first step of our research involved measuring the spread of \gls{staph} \colorbox{PaperColor}{\textcolor{black}{(Paper A)}} and investigating the possibility that inflammatory processes may be similar across individuals or schools \colorbox{ResultColor}{\textcolor{black}{(Result II)}}. With this initial groundwork laid, we did further analyses. We continued to explore other metrics, such as the influence of obesity \colorbox{ResultColor}{\textcolor{black}{(Result I)}} and vitamin D \colorbox{PaperColor}{\textcolor{black}{(Paper B)}} among other measures. 

%Additionally, we used machine learning models to measure social influence on obesity \colorbox{ResultColor}{\textcolor{black}{(Result III)}}. We also reported medicine usage in this population and how friends can influence this behavior \colorbox{ResultColor}{\textcolor{black}{(Result IV)}}. We also presented a tutorial and proof of concept on how to apply \gls{sna} in prescriptions and drug interactions \colorbox{PaperColor}{\textcolor{black}{(Paper C)}}.






%Throughout the process, I aimed to provide guidelines regarding the data cleaning process, we applied \gls{bcnf} to the data to save valuable time. 


% S - Who and What
% M - By how much
% A - How?
% R - Why?





\vspace{0.90\baselineskip}

