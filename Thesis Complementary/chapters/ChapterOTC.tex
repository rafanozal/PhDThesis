%*****************************************
\chapter{OTC Medication}
\label{ch:otc}
%*****************************************

%*****************************************
\section{Introduction}
%*****************************************

\gls{otc} medicine \cite{Algarni2021} refers to a type of medication that is sold without the need for a prescription and can be acquired even outside of pharmaceutical outlets such as supermarkets. These types of medicine have a low dosage of the active agent and are used to alleviate common symptoms such as pain, fever, cough, allergies, and digestive issues.

The OTC medicine market is substantial. Worldwide, there was an estimated spend of 3.68 billion GBP (adjusted) between 2013 to 2019 on codeine-based products alone \cite{Richards2022}. In the US in 2015, there was a spend of 32 billion USD on OTCs with an average of nearly 350 USD per household per year \cite{Basch2016}. Worth mentioning that there are specific marketing strategies to make OTC appealing to children under 6, with all products indicating in different typography the type flavoring, having cases such as 5\% of the cough syrups being chocolate flavored \cite{Basch2016}. \gls{otc} are fairly safe when used in moderation and responsibly, but the accessibility and appeal of these medications also open the door to a variety of potential misuse; for example, from 2010 to 2013, 43\% of the emergency visits in children under 6 related to medicine exposure were due to OTC overdose \cite{Basch2016}. In recent years, the effectiveness of OTCs for the general public has been questioned \cite{McCoul2019} due to the increase in their misuse. About 16\% of the population misuse these medicines, about 2\% abuse them, and up to 7\% recognize addiction \cite{Algarni2021, ijerph18115530, PolandSelfMedicate, Lein2023}. In more extreme cases, other in theory non-OTC medicines abuse, such as antibiotics, becoming de-facto OTC has led Greece to have one of the most antimicrobial resistance rates in Europe \cite{Lionis2014}.

Common reasons for its use are the false belief that if taken more the symptoms will go away faster, or to the contrary, for self-harming purposes \cite{Algarni2021}. In Norway \cite{Lorentzen2018}, teenagers report using these drugs mostly for physical pain, to alleviate stress and fatigue, due to familiar conflicts, and to appear socially successful; while the common user is a person with constant pain, binge drinking, having school problems, bad sleeping habits, lower life ambitions and high spare time. About 55\% of \gls{otc} medicines are sold in non-pharmaceutical outlets \cite{Lorentzen2018}. A study in the Chinese market \cite{Qian2020} showed that despite the increase in marketing budget expended by pharmaceutical, the inclusion, and diffusion of new OTCs is very slow and that the mouth-to-mouth recommendation of peers seems to be much more effective. Witch begs the question of whether or not this type of influence can be seen in our teenage population.


%*****************************************
\section{Principles and direct adverse effects}
%*****************************************

\subsection{Analgesics}

Analgesics are a class of drugs commonly known as painkillers. They reduce or block the perception of pain signals in the brain. \gls{nsaids} usually overlap their effect with analgesic medicine but will be discussed in the antiinflammatories section.

The most common active component of analgesics is paracetamol (acetaminophen) \cite{Brune2014}. The majority of acetaminophen is metabolized safely by conjugation with glucuronic acid (glucuronidation) and sulfate (sulfation) in the liver. These reactions produce non-toxic, water-soluble metabolites that are easily excreted by the kidneys.

About 2\% of it transforms into \gls{napqi}, which remains in the liver for a longer time before it also moves into the kidneys. \gls{napqi} is toxic. When taken in a low amount, NAPQI is quickly detoxified in the liver by conjugation with glutathione, a powerful antioxidant found in the liver. The resulting non-toxic compounds are then excreted via the kidneys. Problems arise when analgesics are taken in higher doses or at a higher frequency, the liver may become overwhelmed and unable to detoxify all of the \gls{napqi}. When glutathione is depleted, NAPQI accumulates in the liver, binds to cellular proteins, and causes oxidative stress which leads to hepatocyte (liver cell) death and can result in acute liver failure

In contrast, non-OTC analgesics are usually opioids such as morphine. They work by binding to specific receptors in the brain and spinal cord called opioid receptors \cite{Schulz2004}. Activating these receptors blocks the transmission of the most severe pain signals. However, they are at risk of addiction and their use is severely restricted.

\subsection{Antihistamines}

Antihistamines are a class of medications that are commonly used to treat allergic rhinitis and allergic-like symptoms. They work by blocking histamine, hence their name. Histamine is released when allergens bind to mast-cell-bound \gls{ige} antibody sites \cite{Vardanyan2016}. This is commonly known as an allergy and is the mechanism of bronchial smooth muscle contraction, urinary bladder contractions, vasodilation, visceral hypersensitivity, itch perception, urticaria, sneezing, hyper-secretion from glandular tissue, and  nasal congestion due to vascular engorgement.


Different antihistamine medications can cause both vasoconstriction and vasodilation. In general, vasodilation may be more effective at relieving symptoms such as congestion and mucus production, while vasoconstriction may be more effective at reducing inflammation and swelling.

Histamine binds to histamine receptors, of which there are four types, but the two most relevant to antihistamine medications are H1 and H2 receptor blockers. First-generation H1 antihistamines can block the neurotransmitter acetylcholine. Acetylcholine, among its many functions, plays a role in the constriction of blood vessels. When acetylcholine's action is blocked, this can lead to a relative decrease in vascular tone (the tension of blood vessel walls), which may result in vasodilation. Second-generation H1 antihistamines have a much lower affinity for cholinergic receptors and therefore have fewer anticholinergic (or sedative) effects. H2 Antihistamines are commonly used to treat excess stomach acid. H2 receptors are found in the stomach lining and are responsible for stimulating acid secretion. However, H2 receptors are also present in the blood vessels and contribute to vasodilation. By blocking H2 receptors, H2 antihistamines can reduce acid secretion and also decrease vasodilation, thus increasing vasoconstriction.

Both cases can be detrimental. Vasoconstriction can increase blood pressure in the heart and reduce blood flow in the kidneys. For patients with decreased kidney functionality or hypertension, this would be dangerous. Vasodilation increases blood flow, which is generally beneficial for normal kidney function. However, it is detrimental in cases when the patient is incapable of filtering waste, and excess fluid from the blood would lead to a buildup of toxins and fluid in the body. Lastly, worth mentioning that the first generation of antihistamines developed during the 1930s have a detrimental flaw and they tend to cross the brain-blood barrier \cite{Walsh2005}. The blood-brain barrier is a specialized system of blood vessels that helps to protect the brain from harmful substances and toxins.

\subsection{Antiinflammatories}

\gls{nsaids} drugs such as ibuprofen work by inhibiting the production of prostaglandins. Prostaglandins cause blood vessels to dilate, which can increase blood flow to the affected area and cause redness and swelling. They also sensitize nerve endings to pain, which can cause pain and discomfort. NSAIDs work by inhibiting the activity of an enzyme called \gls{cox}, (subdivided into COX1 and COX2) which is responsible for the production of \gls{pg} \cite{Faki2021}. These are discussed in chapter \ref{arcachonidacidsPRO} when we talked about how the inflammation process needs to pass from a pro-inflammatory state into an anti-inflammatory state.

NSAIDs can be divided broadly into three categories, COX1 inhibitors, COX2 inhibitors, and both COX1 and COX2 inhibitors. Prostaglandins PGE2 and PGI2 participate in the synthesis of protective mucus and gastric flow, which is why a normal side effect of COX1 inhibitors is gastrointestinal bleeding and ulcers \cite{Faki2021}. COX1 also participates in the production of thromboxane which promotes platelet aggregation, which is why antiinflammatories such as aspirins can cause an increase in bleeding. On the other hand, prostaglandins PGI2 and PGH2 are vasodilators and share COX precursors with thromboxane which is also a vasoconstrictor. Simply put, there must be an equilibrium between COX1 (vasoconstrictor) and COX2 (vasodilator). NSAIDs that inhibit COX2 increase blood pressure which can lead to heart infartion \cite{Faki2021}. These side effects can be mitigated with the usage of prostaglandin analog medicines.

\subsection{Cough syrups}

Cough syrups are used to stop unwanted coughing which may cause discomfort, or even physical injuries, in the upper respiratory tract. Their mechanism of action is not fully understood but their effects are accomplished by reducing the signaling between the laryngeal nerves and vagus nerve \cite{Bardal2011}.

The three main antitussive components are codeine, \gls{dxm}, and benzonatate \cite{Beharry2016}. Codeine breaks down into codeine-6-glucoronide and morphine, which stimulates the µ-opioid receptors \cite{Schulz2004} causing euphoria, constipation, and also cough suppression. \gls{dxm} in the principal component in \gls{otc} medicines, and it has similar side effects of codeine but in the form of hallucinogenics, and in particular as a dissociative. Benzonatate is a non-narcotic drug that works as a local anesthetic in the whole respiratory tract. 

\subsection{Laxatives}

The main component of these drugs is loperamide. Loperamide works by binding to peripheral µ-opioid receptors in the gastrointestinal tract \cite{Malinky2021}. It inhibits the release of acetylcholine and other neurotransmitters that stimulate the contractions of the intestinal wall. This leads to a reduction in peristalsis, or the wave-like contractions of the smooth muscle in the intestinal wall which slows down the passage of stool and reduces the frequency and urgency of bowel movements. This also allows for the gastrointestinal tract to have more time absorbing fluids, increasing the hardness of the fecal matter.

Secondarily, loperamide reduces the sensitivity of the recto-anal inhibitory reflex and increases internal anal sphincter tone; which translates into being able to hold the fecal content inside the rectum much better \cite{Musial1992}.
 
%*****************************************
\section{Adverse drug interactions}
%*****************************************

\subsection{Anticoagulant}

A classical example of adverse drug interaction is Warfarin.  Warfarin is an anticoagulant medication that is used to prevent blood clots from forming or from growing larger. NSAIDs and aspirin also have an anticoagulant effect by inhibiting platelet aggregation, although through different mechanisms than warfarin.

Taking warfarin with these medications is discouraged because their additive anticoagulant effect can lead to a significantly increased risk of bleeding because both the platelet function and the clotting factor production are impaired. This includes gastrointestinal bleeding which can be caused by NSAID irritation in the gut, and exacerbated by warfarin.

\subsection{Antidepresants}

\gls{maois} are a class of antidepressant drugs that work by inhibiting the activity of one or both monoamine oxidase enzymes (MAO-A and MAO-B) \cite{Edinoff2022}. These enzymes are responsible for breaking down neurotransmitters such as serotonin in the brain. By inhibiting these enzymes, MAOIs increase the levels of these neurotransmitters, which can help improve mood and reduce symptoms of depression.

Many over-the-counter cold and cough medications contain sympathomimetic amines, such as pseudoephedrine, phenylephrine, or ephedrine. These substances act as decongestants by constricting blood vessels in the nasal passages.  When taken with MAOIs, the breakdown of these sympathomimetic amines is inhibited, leading to their increased levels and prolonged action. The danger from this is an increase, or longer, vasoconstriction spike.

Furthermore, dextromethorphan, a cough syrup, can increase serotonin levels. When combined with MAOIs, the risk of accumulation of serotonin in the brain is increased; which may lead to confusion, agitation, muscle twitching, or sweating; an effect often sought if someone tries to use misused cough syrup as a recreational drug, but which can derive into seizures, irregular heartbeat, and unconsciousness.

\subsection{Blood pressure regulators}

 As seen in the introduction, NSAIDs of the COX2 can either nullify or enhance other blood pressure medications leading to heart infartation.

\subsection{Other non-OTC supplements}

Saint Jonh's Worth is a popular herbal remedy sold in parapharmacies that has a moderate effect as an antidepressant \cite{Linde2008}. However, what at first glance is just an innocent tea drink, interacts severely with many medications by modifying the proper xenobiotic pathways in the liver (mainly CYP3A4 and CYP2C9 enzymes) \cite{Komoroski2004}.

This can lead to death, by suppressing anticoagulants, such as warfarin mentioned above, anti-HIV medication, and Digoxin which is a medication used to threaten mainly atrial fibrillation in the heart. But can also have other detrimental side effects such as making other medications stop working, such as hormonal contraceptives which lead to pregnancy, Xanax which leads to anxiety and panic attacks, or several antidepressants which may lead to increased depressive disorders.

%*****************************************
\section{Common misuses}
%*****************************************

Analgesics are not used as recreational drugs directly but are used in combination with other substances such as alcohol or cannabinoids \cite{otcAbuse2020}.

The effects of drug abuse related to antihistamines have a huge variety, but as a general rule, they are used as vasodilators, which promote a calming and sedating effect. They can enhance the effect of other substances such as making opioids more hallucinogenic, especially the first generation \cite{otcAbuse2020}.

Anininflamatories are vasoconstrictors and can be used as stimulants, which can also induce psychotic symptoms, paranoia, and visual hallucinations. \cite{otcAbuse2020}

Cough syrups act on the peripheral nervous system, the main side effect of these drugs is reduced consciousness in the form of drowsiness. In higher dosages, it can lead to hallucinations, paranoia, perceptual distortions, delusional beliefs, ataxia, and out-of-body experiences \cite{otcAbuse2020}.

Similar to antitussive medications, antidiarrheals relax the peripheral nervous system. Thus, these drugs are used also as opioids, to alleviate symptoms of opioid withdrawal, or as a psychoactive. \cite{otcAbuse2020}

%*****************************************
\section{ Ethical considerations}
%*****************************************

\subsection{ Legal framework }

There are plenty of different regulations when it comes to OTCs \cite{LpezVila2023}. Within the EU alone, 16 of the 30 European countries allow selling OTCs in non-pharmacy outlets; in 13 countries, the dosage is restricted and the medicine is secure in lockers, and in others purchasing them online without restriction is allowed. Greece or Lithuania have total freedom of sales, whereas Hungary or Poland only allows pharmacies, and the pharmacies must be owned by, at least 51\%, the pharmacist and not by a big corporation. Pharmacies in Denmark, Luxembourg, Slovenia, and Finland are owned by the state. In Iceland and Norway, there was a deregulation of the ownership to promote competition, but now there is a de-facto oligopoly where two and three pharmacy groups control about 90\% of the market. In Japan, pharmaceutical access is supposed to control the abuse of individuals and ensure that purchases are done in moderation but this regulation is rarely followed by pharmacies \cite{Ino2022}; and in many other places, while high dosages such as Ibuprophen 1000mg is restricted, nothing keep a person from taking 5 times a dosage of 200mg instead.

\subsection{ Right to Health }

In the previous paragraph, we can see that there is not a unique approach to these medicines. On the one hand, people should be able to purchase cheap medicines whether they need to, but this leads to misuse or addiction. On the other hand, harsh regulation would prevent people's access to medicine or strain the medical system with an unnecessary burden.

If the market is liberated, manufacturers tend to market their product as seen in the introduction, and rarely do they put effort into educating the public on the possible long-term detrimental side effects \cite{Ino2022}.

\subsection{ Education }

One can think that education is the key to preventing the public from misuse of \gls{otc}. In a questionnaire done to 133 2nd year medical students of a tertiary care hospitals:\textit{ "Most of them did not knew about the contraindications of the OTC drugs and only a few knew about the potential drug-food interactions"} \cite{Prabhuswamy2022}.

The OTC-SOCIOMED initiative (\url{http://www.otcsociomed.uoc.gr/joomla/index.php/key-findings}) concluded \cite{https://doi.org/10.26220/aca.2981} that interventions in GPs prevent overprescribing OTCs and that sale points should be limited to pharmacies; which was further recommended to Greece in particular because, as seen above, they allowed a total liberalization. In contrast, the Congress in the US passed a reform act \cite{Gardiner2021} that: \textit{"allow drug companies a more efficient pathway to marketing OTC products, making OTC drug development more attractive"}, and to prevent misuse pharmaceutical companies should create a database to monitor proper use, but at the same time the act recognize that these companies \textit{"...more likely to accede to industry goals such as rushed reviews and lower standards for approval when it is financially linked to industry. Such organizations cite to reports of increased pressure to meet deadlines and anecdotal evidence of drugs being approved without sufficient scientific support."}

Nurses in school settings suggest proper nurse training, alternative methods such as cold / heat for inflammation and pain, healthy habits, and more importantly, not using OTCs so children stop complaining and can be returned to class quickly. \cite{Wallace2016}

No literature was found regarding the effectiveness of packaging regulation like what is done in tobacco products, such as limiting colors, font size, displaying size effect, and the like.



